\documentclass[MSc, francais, english]{ulthese}
\selectlanguage{english}

\ifxetex\else \usepackage[utf8]{inputenc} \fi

\usepackage{amsmath}          % recommandé pour les mathématiques
%\usepackage{icomma}           % gestion de la virgule dans les nombres

\usepackage{mathpazo}         % texte et mathématiques en Palatino
\usepackage{mathptmx}         % texte et mathématiques en Times

\usepackage{hyperref}
\hypersetup{colorlinks,allcolors=ULlinkcolor}
%% Options de mise en forme du mode français de babel. Consulter la
%% documentation du paquetage babel pour les options disponibles.
%% Désactiver (effacer ou mettre en commentaire) si l'option
%% 'nobabel' est spécifiée au chargement de la classe.
%\frenchbsetup{%
    %CompactItemize=false,         % ne pas compacter les listes
    %ThinSpaceInFrenchNumbers=true % espace fine dans les nombres
%}

%\bibliographystyle{natbib}

\titre{Robot Perception Using LiDAR in Complex Environments}
% \titre{Ceci est un exemple de long titre \\
%   avec saut de ligne manuel}
% \soustitre{Sous-titre le cas échéant}
% \soustitre{Ceci est un exemple de long sous-titre \\
%   avec saut de ligne manuel}
\auteur{Sébastien Michaud}
\programme{Maîtrise en informatique}
\annee{2015}

\begin{document}

\frontmatter

\pagetitre

%\chapter*{Résumé}
\phantomsection\addcontentsline{toc}{chapter}{Résumé}

\begin{otherlanguage*}{francais}

    Pour assurer une navigation sécuritaire et efficace, les robots mobiles reposent grandement sur l'utilisation des capteurs embarqués. L'un des capteurs qui est de plus en plus utilisé pour la navigation des robots est le \emph{Light Detection And Ranging} (LiDAR). Il produit de l'information riche sur la structure tridimensionnelle de l'environnement du robot et ce, indépendamment des conditions lumineuses. Bien que les recherches récentes montrent une amélioration des performances de navigation pour des problèmes contraints comme la conduite urbaine dans les conditions ensoleillées de la Californie, faire face à des environnements non-structurés complexes ou des conditions météorologiques difficiles reste problématique. Dans ce mémoire, nous présentons une analyse de l'influence de telles conditions sur la navigation basée sur les LiDARs. Notre première contribution est d'évaluer comment quatre LiDARs couramment utilisés (Velodyne HDL-32E, SICK LMS151, SICK LMS200 et Hokuyo UTM-30LX-EW) sont affectés par les flocons de neige durant des tempêtes de neige. Nous créons un nouvel ensemble de données en faisant l'acquisition de données avec les quatre capteurs simultanément durant six précipitations de neige. Une analyse statistique de ces ensembles de données indique que leurs mesures peuvent être modélisées de manière probabiliste, permettant l'utilisation d'un système bayésien pour améliorer la robustesse. De plus, nous observons l'évolution temporelle de l'impact de la neige durant ces précipitations, et caractérisons la sensibilité de chaque capteur. Finalement, nous concluons que les précipitations de neige ont peu d'influence au-delà de \SI{10}{\meter}. Notre seconde contribution est d'évaluer l'impact de structures tridimensionnelles complexes présentes en forêt sur les performances des algorithmes de navigation. Étant donné que l'habileté à reconnaître des endroits visités est utile à plusieurs problèmes de navigation, nous avons choisi d'effectuer nos expériences avec un algorithme de l'état de l'art en reconnaissance d'endroits. Nous avons acquis des données dans un environnement extérieur structuré et en forêt ainsi qu'avec deux LiDARs (LMS151, HDL-32E). Cela permet d'évaluer l'influence de l'environnement sur les performances de reconnaissance d'endroits, mais aussi celle du capteur choisi. Notre hypothèse est que le plus proche deux balayages laser sont l'un de l'autre, le plus haut le score (i.e. la croyance que ces balayages laser proviennent du même endroit) sera, mais modulé par le niveau de complexité de l'environnement. Nos expériences confirment que la forêt, avec ses réseaux de branches compliqués et son feuillage, produit plus de données aberrantes et induit une chute plus rapide des scores en fonction de la distance que pour les environnements structurés, présent entre des bâtiments. De façon similaire, les scores pour le Velodyne chutes légèrement plus rapidement en fonction de la distance que pour le SICK. Notre conclusion finale est que, les environnements complexes étudiés impactent négativement les performances de navigation basée sur les LiDARs, ce qui devrait être considéré pour développer des algorithmes de navigation robustes.


\end{otherlanguage*}

%\chapter*{Abstract} \phantomsection\addcontentsline{toc}{chapter}{Abstract} 

\begin{otherlanguage*}{english}

    To ensure safe and efficient navigation, mobile robots heavily rely on their ability to use on-board sensors to perceive and model their environment. One such sensor increasingly used for robot navigation the \gls*{lidar}, which prove to provided rich information about the three-dimensional structure of the robot surroundings, independent of illumination conditions. Although recent research showed improvement in \gls*{lidar}-based navigation performance for constraint problems such as urban driving in sunny California, dealing with challenging conditions such as unstructured environments or difficult weather conditions remain challenging. In this thesis, we present an analysis of the influence of such complex conditions on \gls*{lidar}-based navigation.

    Our first contribution is to evaluate how four commonly-used \gls*{lidar}s (Velodyne HDL-32E, SICK LMS151, SICK LMS200 and Hokuyo UTM-30LX-EW) are affected by snowflakes during snowstorms. We created a novel dataset by acquiring data from the four sensors simultaneously during 6 snowfalls. Statistical analysis of this dataset indicated that these sensor measurements can be modeled in a probabilistic manner, allowing the use of a Bayesian framework to improve robustness. Moreover, we were able to observe the temporal evolution of the impact of the falling snow during these snowstorms, and characterized the sensitivity of each device. Finally, we concluded that the falling snow had little impact beyond a range of \SI{10}{\meter}.

    Our second contribution is to evaluate the impact of complexity of \gls*{3d} structures in an environment on a navigation algorithm performance. Because the ability to recognize previously-visited places is useful for solving many navigation problems, we choose to perform our experiments using a state-of-the-art place recognition algorithm. Using the SICK LMS151, we acquired a dataset in an environment similar to those used in the original place recognition paper and a dataset in a forest for comparison. We also reproduced the forest dataset using the Velodyne HDL-32E, in order to determine if the sensor influences performance. Analysis of the results was performed using the output of the original place recognition algorithm, which represented the belief that two input scans originate from the same place. Our hypothesis was that the closer two scans were acquired from each other, the higher we expected the score to be, but modulated by the level of complexity of the environments. Our experiments confirmed indeed that the forests, with their intricate network of branches and foliage, produced more outliers and induced place recognition scores to decrease more quickly with distance than for open, structured environment, found between building. Similarly, the scores for the Velodyne decrease slighly faster with distance for the Velodyne than for the SICK. 

\end{otherlanguage*} 

%\cleardoublepage

\tableofcontents
\cleardoublepage

\listoftables
\cleardoublepage

\listoffigures
\cleardoublepage

%\dedicace{Dédicace si désiré}
%\cleardoublepage

%\epigraphe{Texte de l'épigraphe}{Source ou auteur}
%\cleardoublepage

\chapter*{Remerciements}         % ne pas numéroter
\phantomsection\addcontentsline{toc}{chapter}{Remerciements} % inclure dans TdM

Thank you !


\mainmatter

\chapter*{Introduction}         % ne pas numéroter
\phantomsection\addcontentsline{toc}{chapter}{Introduction} % inclure dans TdM

Une thèse ou un mémoire devrait normalement débuter par une
introduction. Celle-ci est traitée comme un chapitre normal, sauf
qu'elle n'est pas numérotée.

\chapter{LiDAR Characterization}

\section{Introduction}
\label{sec:chap1_intro}
\begin{itemize}
    \item What is a LiDAR (Useful for more precise,)
    \item How it works
    \item Why is it useful in robotics (visual vs geometrical info)
\end{itemize}

\section{Contribution}
\label{sec:chap1_intro}
Article (Towards Characterizing the Behavior of LiDARs in Snowy Conditions)

\chapter{\chaptwotitle}

\section{Introduction (traversability)}
\label{sec:definition}
\begin{itemize}
    \item What is traversability (there is definition problem)
    \item Measure (qualitative and quantitative)
    \item Classification
\end{itemize}

\section{Machine Learning Basics}
\label{sec:machine_learning}

\section{Experimental setup}
\label{sec:machine_learning}
Describe the Husky


\chapter*{Conclusion}
\phantomsection\addcontentsline{toc}{chapter}{Conclusion}

The objective of this document was to analyze the influence of complex environments on different \gls*{lidar}s used in a context of mobile robotics navigation. Because \gls*{lidar}s can effectively retrieve the three-dimensional structure of an environment, they can advantageously be used for several navigation tasks, such as obstacles avoidance, localization and mapping. The information they provide is complementary to that of other sensors such as cameras, which capture information on the appearance of the surrounding scene. Although the literature on mobile robotic navigation is rich, most existing research deal with indoor or structured outdoor environment. Dealing with complex environments, such as forest or falling snow conditions, remains a important problem. Therefore, analyzing how such complex environments influence the \gls*{lidar}-based navigation can help to develop algorithms that are more robust to changing conditions.

In Chapter~\ref{chap:lidar_snow}, we focused our attention on the impact of falling snow on \gls*{lidar} data. For that matter, we first presented an overview of \gls*{lidar}s functioning, which helped understand what could potentially cause noise in the data. The first issue raised was that, when the spot produced by the laser beam is not completely on a single target, a method of inference of the distance must be chosen and may not be suitable for the application. This is generally caused by object edges of small particles. The second issue raised was that \gls*{lidar}s acquire data points sequentially, which can lead to a distorted representation of the scene when the environment is dynamic. Consequently, the dynamic nature and small size of snowflakes make snowfall a perfect example of challenging condition for \gls*{lidar} acquisition.

To perform our analysis of the impact of falling snow on \gls*{lidar} acquisition, we collected data from four sensors simultaneously, namely the Velodyne HDL-32E, the SICK LMS151, the SICK LMS200 and the Hokuyo UTM-30LX-EW. We acquired data during six snowfalls in a wide variety of conditions. The final dataset used for our analysis contains more than 40 hours of data. 

For our analysis, we first observed how the proportion of \gls*{lidar} returns caused by snowflakes evolved over time. We visually represented the short-term evolution by overlaying the snowflakes according to their spatial arrangement for four subsequent scans. This showed significant quantitative and spatial changes, even over such short time. We observed no pattern in the distribution of snowflakes and believe it can be modelled by a random process. We also illustrated the long-term evolution by showing the proportion of return as function of time, for the whole duration an acquisition. The shape of this curve provides information on the progress of the weather during this period. Additionally, we conducted a comparative analysis of the overall sensitivity to snowflakes of our four sensors. We showed that the SICK LMS200 is the more sensitive with peaks reaching up to \SI{15}{\percent} of echoes caused by falling snow, while other three \gls*{lidar}s never exceeded \SI{1}{\percent}. Beside our temporal analysis, we analyzed how range can affect the probability to trigger a measurement. Based on a histogram, we found that the probability density function is close to a log-normal. One final important finding is that beyond \SI{10}{\meter}, snowflakes no longer seem to trigger measurement.

In Chapter~\ref{chap:slam}, we have conducted our analysis at a higher level, by comparing the performance of a state-of-the-art \gls*{lidar}-based place recognition algorithm in different environments. There are two main reasons for this choice of algorithms . Firstly, the ability of a robot to identify previously visited places allows to solve several other navigation problems, such as multi-session-mapping, kidnapped robot and loop closure in \gls*{slam}. Secondly, the chosen algorithm proved to be successful in indoor and structured outdoor environments, but had not been tested in unstructured environments.

For our comparative analysis, we acquired datasets in two different areas of the Laval University campus using the Husky A200 mobile robot. The first area is our model for a structured outdoor environment, similar to those on which the original algorithm was tested. In this location, the ground is mostly flat and the scanned space contains several man-made objects, such as buildings, stairs and lampposts. The second area, representing our unstructured environment, is in a forest where the ground is uneven in some parts. We also used two different sensors for our acquisitions, namely the SICK LMS151 and the Velodyne HDL-32E. We used the resulting \texttt{Structured-SICK}, \texttt{Unstructured-SICK} and \texttt{Unstructured-Velodyne} datasets to evaluate the impact of both the environment and the sensor choice on the place recognition performance.

%Before performing the comparative analysis itself, we presented some fundamental concepts required to understand the chosen algorithm. This include features keypoints and descriptors, range images and scans comparison techniques. We then proceed to explain the algorithm itself, which first convert point clouds into range images in order to extract NARF features and then use a mixture of \gls*{bow} and features matching to find possible transformations between two scans. The final output of the algorithm is a score computed from the best evaluated transformation betweens two scans, and represent the confidence of the system that those scans originate from the same place. 

We showed the relation between the algorithm output score and the distance separating the input scans. We consider that the closer two scans are from each other, the more likely they are to represent the same place. Therefore, we expected an inverse relation between the score and the distance separating the scans. We also presented the results using binary labels, because practical uses of place recognition generally required pairs of scans to be identified as originating from the same place or not. For that matter, all scans for which the distance between them was below a threshold were considered as belonging to the same place. Similarly, we considered that if the score obtained for a pair of scans was above a threshold, the algorithm labelled the input scans as representing the same place. This labelling also enabled us to identify false positives and calculate the recall for different thresholds combinations. As we explained, false positives must be avoided during \gls*{slam}, because they create false loop closures, which distort the resulting map in a castastrophic manner. 

The presentation of results, as described above, enabled us to make our final observations. As expected, we showed a relation of the form $f(x)=1/x$ between the algorithm output score and the distance between two scans. This was true for our three datasets. However, some outliers were observed for the unstructured environment, whereas this was not the case for the structured environment. This is important when using the discretized labels, because such outliers can result in \gls*{fp} or, if the score threshold is adjusted to avoid them, significantly reduce the recall. Our last observation is that the score decreases more rapidly as a function of the distance between scans in the unstructured environment than in the structured environment. This means that the algorithm becomes sensitive to disturbances and noise as the place recognition range increases more quickly in an unstructured environment than in a structured environment. The same observation applies when comparing the sensors, in which case the algorithm become sensitive faster as the range increases for the Velodyne than the SICK.

To conclude, the goal of this document was to evaluate the influence of complex environments on \gls*{lidar}-based robot navigation. By characterizing \gls*{lidar}s data acquired during snowfalls and by comparing performances of \gls*{lidar}-based place recognition in forest and in structured outdoor environment, we showed that these environments can negatively impact robot navigation. Based on our observations, future research could focus on finding methods to increase robustness to these conditions, for instance by filtering input data. In the case of our place recognition analysis, a better identification and quantification of the sources causing performance decrease could also help the development of better solutions for navigation.


\appendix

\chapter{Title of the Appendix}

Text of the appendix !


%\bibliography{}

\end{document}
