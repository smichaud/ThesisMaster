\documentclass[MSc, francais, english]{ulthese}
\selectlanguage{english}

\ifxetex\else \usepackage[utf8]{inputenc} \fi

\usepackage{amsmath}          % recommandé pour les mathématiques
%\usepackage{icomma}           % gestion de la virgule dans les nombres

\usepackage{mathpazo}         % texte et mathématiques en Palatino
\usepackage{mathptmx}         % texte et mathématiques en Times

\usepackage{hyperref}
\hypersetup{colorlinks,allcolors=ULlinkcolor}
%% Options de mise en forme du mode français de babel. Consulter la
%% documentation du paquetage babel pour les options disponibles.
%% Désactiver (effacer ou mettre en commentaire) si l'option
%% 'nobabel' est spécifiée au chargement de la classe.
%\frenchbsetup{%
    %CompactItemize=false,         % ne pas compacter les listes
    %ThinSpaceInFrenchNumbers=true % espace fine dans les nombres
%}

%\bibliographystyle{natbib}

\titre{Robot Perception Using LiDAR in Complex Environments}
% \titre{Ceci est un exemple de long titre \\
%   avec saut de ligne manuel}
% \soustitre{Sous-titre le cas échéant}
% \soustitre{Ceci est un exemple de long sous-titre \\
%   avec saut de ligne manuel}
\auteur{Sébastien Michaud}
\programme{Maîtrise en informatique}
\annee{2015}

\begin{document}

\frontmatter

\pagetitre

%\chapter*{Résumé}
\phantomsection\addcontentsline{toc}{chapter}{Résumé}

\begin{otherlanguage*}{francais}

    Pour assurer une navigation sécuritaire et efficace, les robots mobiles reposent grandement sur l'utilisation des capteurs embarqués. L'un des capteurs qui est de plus en plus utilisé pour la navigation des robots est le \emph{Light Detection And Ranging} (LiDAR). Il produit de l'information riche sur la structure tridimensionnelle de l'environnement du robot et ce, indépendemment des conditions lumineuses. Bien que les recherches récentes montrent une amélioration des performances de navigation pour des problèmes contraints comme la conduite urbaine dans les conditions ensoleillés de la Californie, faire face à des environnements non-structurés complexes ou des conditions météorologiques difficiles reste problématique. Dans ce mémoire, nous présentons une analyse de l'influence de telles conditions sur la navigation basée sur les LiDARs. Notre première contribution est d'évaluer comment quatre LiDARs couramment utilisés (Velodyne HDL-32E, SICK LMS151, SICK LMS200 and Hokuyo UTM-30LX-EW) sont affectés par les flocons de neige durant des tempêtes de neige. Nous créons un nouvel ensemble de données en faisant l'acquisition de données avec les quatre capteurs simultanément durant six précipitations de neige. Une analyse statistique de ces ensembles de données indique que leurs mesures peuvent être modélisés de manière probalistique, permettant l'utilisation d'un système bayésien pour améliorer la robustesse. De plus, nous observons l'évolution temporelle de l'impact de la neige durant ces précipitations, et caractérisons la sensibilité de chaque capteur. Finalement, nous concluons que les précipitations de neige ont peu d'influence au-delà de \SI{10}{\meter}. Notre seconde contribution est d'évaluer l'impact de structures tridimensionnelles complexes présentes en forêt sur les performances des algorithmes de navigation. Étant donné que l'habileté à reconnaître des endroits visités est utile à plusieurs problèmes de navigation, nous avons choisi d'effectuer nos expériences avec un algorithme de l'état de l'art en reconnaissance d'endroits. Nous avons acquis des données dans un environnement extérieur structuré et en forêt ainsi qu'avec deux LiDARs (LMS151, HDL-32E). Cela permet d'évaluer l'influence de l'environnement sur les performances de reconnaissance d'endroits, mais aussi celle du capteur chosi. Notre hypothèse est que le plus proche deux balayages laser sont l'un de l'autre, le plus haut le score (i.e. la croyance que ces balayages laser proviennent du même endroit) sera, mais modulé par le niveau de complexité de l'environnement. Nos expériences confirment que la forêt, avec ses réseaux de branches compliqués et son feuillage, produit plus de données abérantes et induit un chute plus rapide des scores en fonction de la distance que pour les environnements structurés, présent entre des batîments. De façon similaire, les scores pour le Velodyne chutes légèrement plus rapidement en fonction de la distance que pour le SICK. Notre conclusion finale est que, les environnements complexes étudiés impactent négativement les performances de navigation basée sur les LiDARs, ce qui devrait être considéré pour développer des algorithmes de navigation robustes.


\end{otherlanguage*}

%\chapter*{Abstract}
\phantomsection\addcontentsline{toc}{chapter}{Abstract}

\begin{otherlanguage*}{english}
Text of English abstract.
\end{otherlanguage*}

%\cleardoublepage

\tableofcontents
\cleardoublepage

\listoftables
\cleardoublepage

\listoffigures
\cleardoublepage

%\dedicace{Dédicace si désiré}
%\cleardoublepage

%\epigraphe{Texte de l'épigraphe}{Source ou auteur}
%\cleardoublepage

\chapter*{Remerciements}         % ne pas numéroter
\phantomsection\addcontentsline{toc}{chapter}{Remerciements} % inclure dans TdM

Thank you !


\mainmatter

\chapter*{Introduction} \phantomsection\addcontentsline{toc}{chapter}{Introduction}

Modern robotics has experienced tremendous growth since its inception during the industrial revolution. Whether it is to assist humans in their works, to automate some tasks or to perform dangerous jobs, robots are emerging in a wide variety of applications. Recent technology such as the self-driving car project from Google and the BigDog quadruped robot from Boston Dynamics are feat of engineering that show the evolution of robotics from the not too distant past, when robots could only achieved simple tasks in controlled environments. One of the key elements that allowed for such progress and that is always a hot topic in research, is the ability of the robot to adapt to changing environments. This is especially true for mobile robotics, where not only the surrounding environment can change, but the robot itself can move and must therefore be able to locate itself. This capability highly depend on algorithms that convert the raw inputs of available sensors into a convenient representation of the environment; concept known as robot perception.

A wide range of sensors are available on the market and the choice of these depends primarily on the specific problem to be solved and the available budget. A fairly inexpensive, yet powerful sensor for navigation is the \gls{imu}. Composed of accelerometers, gyroscopes and sometimes magnetometers, it allows to infer position and orientation changes of the robotic platform, as well as gravitational forces. Although this type of sensor can be really precise over short distances, it accumulates error over time, which inevitably leads to a drift on the pose estimation. This problem can easily be solved by fusing \gls{imu} data with those of another sensor for which the error is time independent. Probably due to its ease of use, the \gls{gps} is among the most popular tool for this task. The major drawback of this sensor is that it requires to receive the signal of at least three satellites at all time. These signals can be blocked by building, terrain, dense foliage or other structures, thus causing important positioning errors or possibly no positioning at all. To work around this problem, you can use a map of landmarks representing the environment in which the robot moves. Whether this map is pre-existing or created progressively, the robot have to be equipped with sensors able to recognize these landmarks. Tactile sensor, infrared range finder, \gls{sonar}, \gls{radar} and multiple type of cameras are examples of such sensors.

These sensors capable of reacting to external stimuli are called exteroceptive sensors. In addition to their use for localization, the latter are needed to solve numerous other problems related to navigation: obstacles avoidance, traversability assessment and mapping just to name a few. Arguably, the most widely used of these sensors is the \gls{rgb} camera. In addition to its price generally very low, cameras provides a rich data set on the scene in which the robot operates. Despite these obvious advantages, processing camera data generally rely on very complex algorithms that are often based on strong assumptions. The processing power and time required by those algorithms is not always available. The \gls{lidar} is another sensor that provide rich data about the robot scene. It use laser to measure distances at different angles from the sensor center. Most \gls{lidar}s provide \gls{2d} set of points and can be mounted on a rotating unit to create \gls{3d} data, while other can directly provide such data. \gls{lidar}s are generally more expensive than cameras, but the geometrical information obtained therewith is often complementary to the appearance information provided by the latter. For instance, a \gls{lidar} will faithfully report the structure of a white wall, while the lack of appearance will make it impossible for the camera to infer such information. However, a camera would be able to locate itself using appearance information of a poster on the wall, while the geometrical information of a flat wall could hardly help for the localisation task. Another important difference between those sensors is that the \gls{lidar} is an active sensor that can be used day and night and which is mostly unaffected by lightning condition, while it is very difficult for the passive camera to deal with changes in lightning condition \todo{rephrase that last sentence}.

\gls{rgb} cameras and \gls{lidar}s are both really well suited for multiple task related to mobile robotics navigation. Although there is still considerable research on the use of cameras, the literature for this sensor is significantly larger than for the more recent and expensive \gls{lidar}s. This is one reason that motivates us to better assess the potential of \gls{lidar}s for this task. Existing works for that sensor mainly deal with simpler situations such as structured indoor or semi-structured city environments. We will therefore focus our attention on more challenging environments such as falling snow conditions or highly unstructured outdoor area. \todo{rephrase and add references} Chapter 1 is about \gls{lidar} Characterization, chapter 2 is about traversability and the third chapter is about place recognition.

\chapter{LiDAR Characterization}

\section{Introduction}
\label{sec:chap1_intro}
\begin{itemize}
    \item What is a LiDAR (Useful for more precise,)
    \item How it works
    \item Why is it useful in robotics (visual vs geometrical info)
\end{itemize}

\section{Contribution}
\label{sec:chap1_intro}
Article (Towards Characterizing the Behavior of LiDARs in Snowy Conditions)

\chapter{My Second Chapter}

Some other random stuff

\chapter*{Conclusion}
\phantomsection\addcontentsline{toc}{chapter}{Conclusion}

The conclusion does not have number !


\appendix

\chapter{Title of the Appendix}

Text of the appendix !


%\bibliography{}

\end{document}
