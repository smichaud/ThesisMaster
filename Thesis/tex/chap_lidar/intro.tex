\section{Introduction}
\label{sec:chap_lidar_intro}

An internal representation of the environment is essential for robots to perform the various tasks for which they are designed. If such representation is not provided beforehand, which is often the case, it will be created using sensors available on the robot. Unfortunately, each sensor acquires a specific type of data in a limited measurement interval. In addition, either because of the sensor itself or because of the acquisition environment, the data obtained are always noisy. While this can be of little influence for some simple problems, ignoring these problems can cause serious misunderstanding of the scene and lead to the failure of the tasks, potentially causing damage to the robot or injuring humans. As we will see in this chapter, \gls*{lidar}s enable us to assess the three-dimensional structure the environment, but they are especially noisy when measuring dynamic objects, small structures or object edges. Forested area and falling snow conditions are good examples of such challenging environments for \gls*{lidar}s. Characterizing how \gls*{lidar}s will react in those conditions will allow us to develop more robust and versatile algorithms.

In this chapter, we will first introduce the basics of \gls*{lidar}s operation (in section ~\ref{sec:chap_lidar_basics}) to better understand why they are affected by small structures. We will follow up with our main contribution, that is to provide a characterization of the behavior of four well-known \gls*{lidar}s in snowy conditions. Through an extensive empirical study performed on a novel dataset captured under varying degrees of snowfall, we evaluate how much these LiDARs are sensitive---or not---to falling snow. We show that recent advances in sensor design have increased their robustness even to significant snowfall. Section~\ref{sec:chap_lidar_data_acquisition} describe how data acquisition was performed, section~\ref{sec:chap_lidar_temporal} present a temporal analysis of the data and section~\ref{sec:chap_lidar_histo} describe the distribution of snowflake echoes as a function of range before we conclude in section~\ref{sec:chap_lidar_conclu}. 

