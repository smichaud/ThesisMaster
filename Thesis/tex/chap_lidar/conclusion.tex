\section{Discussion and Conclusion}
\label{sec:chap_lidar_conclu}

In this chapter, we explored the impact of falling snow on the usability of 4 commonly deployed LiDARs. To this end, we first presented an overview of \gls*{lidar}s functioning and possible causes of noise in the measurements. As explained, the small size and dynamic nature of snowflakes make snowstorms perfect examples of challenging condition when using \gls*{lidar}s.

For our experiments, we collected data during 6 snowstorms in the winter of 2015. Upon analysis, we found that the SICK LMS200 was the most sensitive LiDAR, having a peak average rate of up to \SI{15}{\percent} of echoes coming from falling snow. Meanwhile, all three others never exceeded \SI{1}{\percent}. We also presented a simple probabilistic model to take into account the effect of the range on snowflakes interference. Based on a histogram analysis, we concluded that for our experimental setup, this model can be approximated by a log-normal distribution. Most importantly, our data indicate that the impact of snowflakes on LiDAR beyond a range of \SI{10}{\meter} is very limited. 

However, a number of questions remains to explore. For example, as the LiDAR beam travels through the falling snow, its intensity will diminish. Since the maximum range of a LiDAR is heavily related to this beam intensity, we expect the maximum range to be affected during snowstorms. In our setup, we have not witnessed this issue, indicating that this effect probably happens beyond our maximum distance of \SI{20}{\meter}. Another aspect to be investigated is the relationship between the returned intensities and the surface type (ground or snowflakes). Also, because of the shielding effect of the building, very few snowflakes were present at close range; It might be the case that at closer range, a snowflake might be detected at more than one angle, effectively occluding small targets. Moreover, we have not investigated the impact on the measurement noise for the snowy ground surface in the presence of falling snow.
