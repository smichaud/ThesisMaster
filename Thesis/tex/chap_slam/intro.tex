\section{Introduction}
\label{sec:chap_slam_intro}

In order to navigate safely and efficiently in their environments, mobile robots have to be able to solve a multitude of problems. An example of such problems is the ability to determine whether the robot is located in a place it visited before or in a new, unvisited one.  Despite the fact that this question seems relatively elementary, solving the so-called place recognition problem is useful for a wide variety of applications. For instance, multiple robots can cooperate to concurrently build a global map (multi-session mapping) using recognized places as connecting points between the different local maps ~\citep{Howard2004}. The well-known "kidnapped robot" problem, which consists in determining if the robot has been carried to an arbitrary location, can also be solved using a place recognition algorithm. Because such algorithms do not rely on odometry and allow a robot to locate itself relative to all previously visited places, it is possible to detect this kind of unpredictable change of location. Finally, place recognition algorithms are useful to perform \gls*{slam}. The most obvious use is for a topological representation, where the map consists of places and links between them, but it is also essential for loop closure (often referred as the "front-end") when using a metric representation. 


In the previous chapter, we analyzed the influence of an environment with snowy condition on the \gls*{lidar} data. Following the same idea, we are now interested identifying the impact of challenging environments when performing place recognition using \gls*{lidar}. More precisely, we want to evaluate how unstructured environments, such as forests, influence navigation algorithms performance. To this end, we chose to use a state-of-the-art \gls*{lidar}-based place recognition algorithm developed by~\citet{Steder2011b}. Their algorithm proved to be successful in structured indoor and semi-structured outdoor environments. In our experiments, we produced our own datasets in forests, but also in semi-structured conditions for comparison purposes. Besides the influence of the type of environment, we are also interested in the impact of the sensor used and data associated with it. For this analysis, we used two sensors, namely the SICK LMS151 and the Velodyne HDL-32E. The chapter is divided as follows, Section~\ref{sec:chap_slam_data_acquisition} details where and how the datasets were produced, as well as the resulting data. Thereafter, fundamental concepts related to the place recognition algorithm and the algorithm itself will be presented in Section~\ref{sec:chap_slam_algo}. Finally, the results of the comparative analysis will be presented in Section~\ref{sec:chap_slam_results} before we conclude.


