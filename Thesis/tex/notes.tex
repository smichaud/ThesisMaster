\chapter{Personnal Notes}

Structure for intro
\begin{itemize}
    \item What the topic is about
    \item Niche (why there is need for research on your topic)
    \item Introduce the current research (hypothesis, research question)
\end{itemize}

Elements of the intro
\begin{enumerate}
    \item state the general topic and give some background
    \item provide a review of the literature related to the topic
    \item define the terms and scope of the topic
    \item outline the current situation
    \item evaluate the current situation (advantages/ disadvantages) and identify the gap
    \item identify the importance of the proposed research
    \item state the research problem/ questions
    \item state the research aims and/or research objectives
    \item state the hypotheses
    \item outline the order of information in the thesis
    \item outline the methodology
\end{enumerate}

My idea for intro
\begin{itemize}
    \item Start with stats about robotics growth
    \item Might wanna talk about the roomba and/or google car.
    \item Different field involved (mechanic, electric, software)
    \item State that "robotics" is quite large topic (industrial, service, surgery, military...)
    \item Mobile robotics is more challenging for environments variability/uncertainty
    \item Perception using different sensors is a key component to deal with the environment. (Camera, Global Positioning System (GPS), Light Detection and Ranging (LiDAR) to name a few)
    \item Problem far from being solved
    \item This thesis focus on the mobile robotics in complex environments(unstructured, complex) such as forest or falling snow scenarios.
    \item Many uses for outdoor complex environments (cars driving in snowstorm, forest inventory, search and rescue)
\end{itemize}

Intro sentence example
\begin{itemize}
    \item In recent decades the study of robotics has expanded from a discipline centered on the study of mechatronic devices to a much broader interdisciplinary subject
    \item In the last few decades many robots have been developed, produced and installed mainly for industrial applications. Despite original concerns that robots would replace human labor, and in the process create large unemployment problems, the reverse was that robots become robots extensions.
    \item Service robots which act in environments populated by humans have become very popular in the last few years. A variety of systems exists which act for example in hospitals, office buildings, department stores, and museums. Furthermore, several multi-robot systems have been developed for tasks which can be accomplished more efficiently by a whole team of robots than just by a single robot.  These tasks include surface cleaning, deliveries, and the exploration of unknown terrain. Whenever teams of mobile robots are operating in the same environment their motions have to be coordinated in order to avoid congestions or collisions.  At the same time the robots should perform their navigation tasks in a minimum amount of time. Thus, sophisticated path planning techniques are needed that fulfill these requirements
\end{itemize}


What is methods:
\begin{itemize}
    \item Information to allow the reader to assess the believability of your results.
    \item Information needed by another researcher to replicate your experiment.
    \item Description of your materials, procedure, theory.
    \item Calculations, technique, procedure, equipment, and calibration plots.
    \item Limitations, assumptions, and range of validity.
    \item Desciption of your analystical methods, including reference to any specialized statistical software.
\end{itemize}

