\chapter*{Introduction}
\phantomsection\addcontentsline{toc}{chapter}{Introduction}

Modern robotics has experienced tremendous growth since its inception during the industrial revolution. Whether it is to assist humans in their works, to automate some tasks or to perform dangerous jobs, robots are emerging in a wide variety of applications. Recent technology such as the self-driving car project from Google or the BigDog quadruped robot from Boston Dynamics are feat of engineering that show the evolution of robotics from the not too distant past, when robots could only achieved simple tasks in controlled environments. One of the key elements that allowed for such progress and that is always a hot topic in research, is the ability of the robot to adapt to changing environments. This capability highly depends on algorithms that convert the raw inputs of available sensors into a convenient representation of the environment; concept known as robot perception.

A wide range of sensors are available on the market and the choice of these depends primarily on the specific problem to be solved and the available budget. Arguably, one of the most widely used sensor is the camera. In addition to its price generally very low, cameras provides a rich data set on the scene in which the robot operates. They can be use to solve problems such as object recognition, localisation, mapping, 3D reconstruction and many more. Despite these obvious advantages, processing camera data generally rely on very complex algorithms that often make strong assumptions. The processing power and time required by those algorithms is not always available. 



\begin{itemize}
    \item Ok, easier said then done... As a human you can easily identify a cat in a picture, but researcher struggled for X years to get decent results in this field.
    \item Require a lot of processing power and time
    \item Depends on light and texture/edges (even 3D interpolation depends on it... to white wall will yield pure crap)
\end{itemize}

\begin{itemize}
    \item My work focus on Light Detection and Ranging sensors (LiDAR).
    \item They are kind of similar to camera in the sense that they give exteroceptive information (information about the surrounding of the robot)
    \item Instead of giving appearance information, they give geometrical info.
    \item They are also more recent and less work have been done these.
\end{itemize}
\begin{itemize}
    \item Definition of challenging
    \item Navigation (good solution for indoor/structured env.)
\end{itemize}

The global positioning system (GPS) is among the most popular and provides a good solution for localisation. Despite its ease of use, GPS does not work in all situations
