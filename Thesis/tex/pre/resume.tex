\chapter*{Résumé}
\phantomsection\addcontentsline{toc}{chapter}{Résumé}

\begin{otherlanguage*}{francais}

    Pour assurer une navigation sécuritaire et efficace, les robots mobiles reposent grandement sur l'utilisation des capteurs embarqués. L'un des capteurs qui est de plus en plus utilisé pour la navigation des robots est le \emph{Light Detection And Ranging} (LiDAR). Il produit de l'information riche sur la structure tridimensionnelle de l'environnement du robot et ce, indépendemment des conditions lumineuses. Bien que les recherches récentes montrent une amélioration des performances de navigation pour des problèmes contraints comme la conduite urbaine dans les conditions ensoleillés de la Californie, faire face à des environnements non-structurés complexes ou des conditions météorologiques difficiles reste problématique. Dans ce mémoire, nous présentons une analyse de l'influence de telles conditions sur la navigation basée sur les LiDARs. Notre première contribution est d'évaluer comment quatre LiDARs couramment utilisés (Velodyne HDL-32E, SICK LMS151, SICK LMS200 and Hokuyo UTM-30LX-EW) sont affectés par les flocons de neige durant des tempêtes de neige. Nous créons un nouvel ensemble de données en faisant l'acquisition de données avec les quatre capteurs simultanément durant six précipitations de neige. Une analyse statistique de ces ensembles de données indique que leurs mesures peuvent être modélisés de manière probalistique, permettant l'utilisation d'un système bayésien pour améliorer la robustesse. De plus, nous observons l'évolution temporelle de l'impact de la neige durant ces précipitations, et caractérisons la sensibilité de chaque capteur. Finalement, nous concluons que les précipitations de neige ont peu d'influence au-delà de \SI{10}{\meter}. Notre seconde contribution est d'évaluer l'impact de structures tridimensionnelles complexes présentes en forêt sur les performances des algorithmes de navigation. Étant donné que l'habileté à reconnaître des endroits visités est utile à plusieurs problèmes de navigation, nous avons choisi d'effectuer nos expériences avec un algorithme de l'état de l'art en reconnaissance d'endroits. Nous avons acquis des données dans un environnement extérieur structuré et en forêt ainsi qu'avec deux LiDARs (LMS151, HDL-32E). Cela permet d'évaluer l'influence de l'environnement sur les performances de reconnaissance d'endroits, mais aussi celle du capteur chosi. Notre hypothèse est que le plus proche deux balayages laser sont l'un de l'autre, le plus haut le score (i.e. la croyance que ces balayages laser proviennent du même endroit) sera, mais modulé par le niveau de complexité de l'environnement. Nos expériences confirment que la forêt, avec ses réseaux de branches compliqués et son feuillage, produit plus de données abérantes et induit un chute plus rapide des scores en fonction de la distance que pour les environnements structurés, présent entre des batîments. De façon similaire, les scores pour le Velodyne chutes légèrement plus rapidement en fonction de la distance que pour le SICK. Notre conclusion finale est que, les environnements complexes étudiés impactent négativement les performances de navigation basée sur les LiDARs, ce qui devrait être considéré pour développer des algorithmes de navigation robustes.


\end{otherlanguage*}
