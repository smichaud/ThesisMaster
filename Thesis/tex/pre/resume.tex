\chapter*{Résumé}
\phantomsection\addcontentsline{toc}{chapter}{Résumé}

\begin{otherlanguage*}{francais}

    La navigation sécuritaire et efficace des robots mobiles repose grandement sur l'utilisation des capteurs embarqués. L'un des capteurs qui est de plus en plus utilisé pour cette tâche est le \emph{Light Detection And Ranging} (LiDAR). Bien que les recherches récentes montrent une amélioration des performances de navigation basée sur les LiDARs, faire face à des environnements non structurés complexes ou des conditions météorologiques difficiles reste problématique. Dans ce mémoire, nous présentons une analyse de l'influence de telles conditions sur la navigation basée sur les LiDARs. Notre première contribution est d'évaluer comment les LiDARs sont affectés par les flocons de neige durant les tempêtes de neige. Pour ce faire, nous créons un nouvel ensemble de données en faisant l'acquisition de données durant six précipitations de neige. Une analyse statistique de ces ensembles de données, nous caractérisons la sensibilité de chaque capteur et montrons que les mesures de capteurs peuvent être modélisées de manière probabilistique. Nous montrons aussi que les précipitations de neige ont peu d'influence au-delà de \SI{10}{\meter}. Notre seconde contribution est d'évaluer l'impact de structures tridimensionnelles complexes présentes en forêt sur les performances d'un algorithme de reconnaissance d'endroits. Nous avons acquis des données dans un environnement extérieur structuré et en forêt, ce qui permet d'évaluer l'influence de ces derniers sur les performances de reconnaissance d'endroits. Notre hypothèse est que, plus deux balayages laser sont proches l'un de l'autre, plus la croyance que ceux-ci proviennent du même endroit sera élevée, mais modulé par le niveau de complexité de l'environnement. Nos expériences confirment que la forêt, avec ses réseaux de branches compliqués et son feuillage, produit plus de données aberrantes et induit une chute plus rapide des performances de reconnaissance en fonction de la distance. Notre conclusion finale est que, les environnements complexes étudiés influencent négativement les performances de navigation basée sur les LiDARs, ce qui devrait être considéré pour développer des algorithmes de navigation robustes.


\end{otherlanguage*}
