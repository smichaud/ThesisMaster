\chapter*{Abstract} \phantomsection\addcontentsline{toc}{chapter}{Abstract} 

\begin{otherlanguage*}{english}

    To ensure safe and efficient navigation, mobile robots heavily rely on their ability to use on-board sensors. One such sensor, increasingly used for robot navigation, is the \gls*{lidar}. It provides rich information about the three-dimensional structure of the robot surroundings, independent of illumination conditions. Although recent research showed improvement in \gls*{lidar}-based navigation performance for constrained problems such as urban driving in sunny California, dealing with complex unstructured environments or difficult weather conditions remains problematic. In this thesis, we present an analysis of the influence of such challenging conditions on \gls*{lidar}-based navigation. Our first contribution is to evaluate how four commonly-used \gls*{lidar}s (Velodyne HDL-32E, SICK LMS151, SICK LMS200 and Hokuyo UTM-30LX-EW) are affected by snowflakes during snowstorms. We create a novel dataset by acquiring data from the four sensors simultaneously during six snowfalls. Statistical analysis of this dataset indicates that these sensor measurements can be modeled in a probabilistic manner, allowing the use of a Bayesian framework to improve robustness. Moreover, we observe the temporal evolution of the impact of the falling snow during these snowstorms, and characterize the sensitivity of each device. Finally, we conclude that the falling snow have little impact beyond a range of \SI{10}{\meter}. Our second contribution is to evaluate the impact of complex of three-dimensional structures present in forests on the performance of a navigation algorithm. Because the ability to recognize previously-visited places is useful for solving many navigation problems, we choose to perform our experiments using a state-of-the-art place recognition algorithm. We acquire data in structured outdoor environment and in forest using two \gls*{lidar}s (LMS151, HDL-32E). This allows to evaluate the impact of both the environment and the choice of sensor on the place recognition performance. Our hypothesis is that the closer two scans are acquired from each other, the higher the score (i.e. the belief that the scans originate from the same place) will be, but modulated by the level of complexity of the environments. Our experiments indeed confirm that forests, with their intricate network of branches and foliage, produce more outliers and induce scores to decrease more quickly with distance than for open, structured environment, found between building. Similarly, the scores for the Velodyne decrease slightly faster with distance than for the SICK. Our final conclusion is that falling snow conditions and forest environments negatively impact \gls*{lidar}-based navigation performance, which should be considered to develop robust navigation algorithms.

\end{otherlanguage*} 
