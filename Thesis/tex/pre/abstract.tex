\chapter*{Abstract} \phantomsection\addcontentsline{toc}{chapter}{Abstract} 

\begin{otherlanguage*}{english}

    To ensure safe and efficient navigation, mobile robots heavily rely on their ability to use on-board sensors to perceive and model their environment. One such sensor increasingly used for robot navigation the \gls*{lidar}, which prove to provided rich information about the three-dimensional structure of the robot surroundings, independent of illumination conditions. Although recent research showed improvement in \gls*{lidar}-based navigation performance for constraint problems such as urban driving in sunny California, dealing with challenging conditions such as unstructured environments or difficult weather conditions remain challenging. In this thesis, we present an analysis of the influence of such complex conditions on \gls*{lidar}-based navigation.

    Our first contribution is to evaluate how four commonly-used \gls*{lidar}s (Velodyne HDL-32E, SICK LMS151, SICK LMS200 and Hokuyo UTM-30LX-EW) are affected by snowflakes during snowstorms. We created a novel dataset by acquiring data from the four sensors simultaneously during 6 snowfalls. Statistical analysis of this dataset indicated that these sensor measurements can be modeled in a probabilistic manner, allowing the use of a Bayesian framework to improve robustness. Moreover, we were able to observe the temporal evolution of the impact of the falling snow during these snowstorms, and characterized the sensitivity of each device. Finally, we concluded that the falling snow had little impact beyond a range of \SI{10}{\meter}.

    Our second contribution is to evaluate the impact of complexity of \gls*{3d} structures in an environment on a navigation algorithm performance. Because the ability to recognize previously-visited places is useful for solving many navigation problems, we choose to perform our experiments using a state-of-the-art place recognition algorithm. Using the SICK LMS151, we acquired a dataset in an environment similar to those used in the original place recognition paper and a dataset in a forest for comparison. We also reproduced the forest dataset using the Velodyne HDL-32E, in order to determine if the sensor influences performance. Analysis of the results was performed using the output of the original place recognition algorithm, which represented the belief that two input scans originate from the same place. Our hypothesis was that the closer two scans were acquired from each other, the higher we expected the score to be, but modulated by the level of complexity of the environments. Our experiments confirmed indeed that the forests, with their intricate network of branches and foliage, produced more outliers and induced place recognition scores to decrease more quickly with distance than for open, structured environment, found between building. Similarly, the scores for the Velodyne decrease slighly faster with distance for the Velodyne than for the SICK. 

\end{otherlanguage*} 
