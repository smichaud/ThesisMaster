\chapter*{Abstract} 
\phantomsection\addcontentsline{toc}{chapter}{Abstract}

\begin{otherlanguage*}{english}

To ensure safe and efficient navigation, mobile robots heavily rely on their ability to use on-board sensors. One such sensor, increasingly used for robot navigation, is the \emph{Light Detection And Ranging} (LiDAR). Although recent research showed improvement in LiDAR-based navigation, dealing with complex unstructured environments or difficult weather conditions remains problematic. In this thesis, we present an analysis of the influence of such challenging conditions on LiDAR-based navigation. Our first contribution is to evaluate how LiDARs are affected by snowflakes during snowstorms. To this end, we create a novel dataset by acquiring data during six snowfalls using four sensors simultaneously. Based on statistical analysis of this dataset, we characterized the sensitivity of each device and showed that sensor measurements can be modelled in a probabilistic manner. We also showed that falling snow has little impact beyond a range of \SI{10}{\meter}. Our second contribution is to evaluate the impact of complex of three-dimensional structures, present in forests, on the performance of a LiDAR-based place recognition algorithm. We acquired data in structured outdoor environment and in forest, which allowed evaluating the impact of the environment on the place recognition performance. Our hypothesis was that the closer two scans are acquired from each other, the higher the belief that the scans originate from the same place will be, but modulated by the level of complexity of the environments. Our experiments confirmed that forests, with their intricate network of branches and foliage, produce more outliers and induce recognition performance to decrease more quickly with distance when compared with structured outdoor environment. Our conclusion is that falling snow conditions and forest environments negatively impact LiDAR-based navigation performance, which should be considered to develop robust navigation algorithms.

\end{otherlanguage*}
