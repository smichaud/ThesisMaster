% \iffalse meta-comment
%
% Copyright (C) 2012 Universite Laval
%
% This file may be distributed and/or modified under the conditions
% of the LaTeX Project Public License, either version 1.3c of this
% license or (at your option) any later version. The latest version
% of this license is in:
%
%   http://www.latex-project.org/lppl.txt
%
% and version 1.3c or later is part of all distributions of LaTeX
% version 2006/05/20 or later.
%
% This work has the LPPL maintenance status `maintained'.
%
% The Current Maintainer of this work is Universite Laval
% <ulthese-dev@bibl.ulaval.ca>.
%
% This work consists of the files ulthese.dtx and ulthese.ins
% and the derived files listed in the README file.
%
% \fi
%
% \iffalse
%<*dtx>
\ProvidesFile{ulthese.dtx}
%</dtx>
%<class>\NeedsTeXFormat{LaTeX2e}[2009/09/24]
%<class>\ProvidesClass{ulthese}%
%<*class>
  [2014/05/23 v3.1 Classe pour les theses et memoires de l'Universite Laval]
%</class>
%<*driver>
\documentclass[11pt]{ltxdoc}
  \usepackage[utf8]{inputenc}
  \usepackage[T1]{fontenc}
  \usepackage{natbib}
  \usepackage[francais]{babel}
  \usepackage[autolanguage]{numprint}
  \usepackage{mathpazo}
  \usepackage[scaled=0.92]{helvet}
  \usepackage{color,booktabs,metalogo}
  \EnableCrossrefs
  \CodelineIndex
  \RecordChanges

  \definecolor{link}{rgb}{0,0.4,0.6}   % ~RoyalBlue de dvips
  \definecolor{url}{rgb}{0.6,0,0}      % rouge-brun
  \definecolor{citation}{rgb}{0,0.5,0} % vert foncé

  \usepackage{hyperref}
  \hypersetup{colorlinks, linktocpage,
    urlcolor=url, linkcolor=link, citecolor=citation,
    bookmarksopen, bookmarksnumbered, bookmarksdepth=subsubsection.
    pdftitle={Manuel de référence de la classe ulthese},
    pdfauthor={Faculté des études supérieures et postdoctorales}}
  \renewcommand{\appendixautorefname}{annexe}
  \renewcommand{\tableautorefname}{tableau}

  \frenchbsetup{%
    CompactItemize=false,         % ne pas compacter les listes
    ThinSpaceInFrenchNumbers=true % espace fine dans les nombres
  }
  \addto\captionsfrench{\def\tablename{{\scshape Tab.}}}

\begin{document}
  \DocInput{ulthese.dtx}
\end{document}
%</driver>
% \fi
% \CheckSum{690}
% \DoNotIndex{\',\^,\`,\ ,\ae}
% \DoNotIndex{\RequirePackage,\ExecuteOptions,\ifthenelse,\ProcessOptions}
% \DoNotIndex{\newcommand,\newcommand*,\newboolean}
% \DoNotIndex{\setlength,\setboolean}
% \changes{3.1}{2014-05-23}{Prise en charge de la maîtrise en bidiplomation}
% \changes{3.0a}{2014-03-24}{Modifications et corrections à la
% documentation, notamment relativement à la configuration de \textsf{natbib}}
% \changes{3.0}{2014-01-06}{Déclaration du grade en option de la classe. Moteur {\XeLaTeX} supporté; ajout de l'option \texttt{nobabel}}
% \changes{2.1}{2013-01-16}{Utilisation transparente de la police Helvetica pour la page titre}
% \changes{2.0}{2013-01-13}{Traitement automatique des longs titres}
% \changes{1.0b}{2012-11-11}{Ajouts et corrections mineures dans la documentation}
% \changes{1.0a}{2012-10-17}{Précisions dans la documentation}
% \changes{1.0}{2012-09-30}{Version initiale}
% \GetFileInfo{ulthese.dtx}
% \title{\textsf{ulthese}: une classe pour les thèses et mémoires de
%   l'Université Laval\thanks{Ce document décrit la classe
%   \textsf{ulthese}~\fileversion, datée du \filedate.}}
% \author{Faculté des études supérieures et postdoctorales\thanks{%
%   Cette classe et sa documentation ont été rédigées par Vincent
%   Goulet~(Faculté des sciences et de génie) avec la collaboration de
%   Koassi D'Almeida~(Faculté des études supérieures et postdoctorales) et
%   Pierre Lasou~(Bibliothèque).}}
% \maketitle
% \tableofcontents
%
% \section{Introduction}
%
% La classe \textsf{ulthese} permet de composer avec {\LaTeX} ou
% {\XeLaTeX} des thèses et mémoires immédiatement conformes aux règles
% générales de présentation matérielle de la Faculté des études
% supérieures et postdoctorales (FESP) de l'Université Laval. Ces
% règles définissent principalement la présentation de la page titre
% des thèses et mémoires ainsi que la disposition du texte sur la
% page. La classe en elle-même est donc relativement simple.
%
% Cependant, la classe \textsf{ulthese} est basée sur la classe
% \textsf{memoir}, une extension de la classe standard \textsf{book}
% facilitant à plusieurs égards la préparation de documents d'allure
% professionnelle dans {\LaTeX}. La classe \textsf{memoir} est très
% configurable et incorpore d'office plus de 30 des paquetages
% (\emph{packages}) les plus populaires\footnote{%
%   Consulter la section~18.24 de la documentation de \textsf{memoir}
%   pour la liste ou encore le fichier journal (\emph{log}) de la
%   compilation d'un document utilisant la classe \textsf{ulthese}.}. %
% L'intégralité des fonctionnalités de \textsf{memoir} est disponible
% dans \textsf{ulthese}.
%
% La classe \textsf{memoir} fait maintenant partie des distributions
% {\LaTeX} modernes; elle devrait donc être installée et disponible sur
% votre système. La classe est livrée avec une documentation exhaustive:
% le manuel d'instructions fait près de 600~pages! Il peut être utile de
% s'y référer de temps à autre pour réaliser une mise en page
% particulière. Rechercher sur votre système le fichier |memman.pdf| ou
% le %
% \href{http://mirrors.ctan.org/macros/latex/contrib/memoir/memman.pdf}{%
%   consulter en ligne} %
% sur le site %
% \href{http://www.ctan.org/}{%
%   \emph{Comprehensive R Archive Network} (CTAN)}.
%
%
% \section{Installation}
%
% La classe est distribuée sous forme d'une archive |ulthese.zip| via
% le réseau de sites CTAN:
% \begin{quote}
%   \url{http://www.ctan.org/pkg/ulthese}
% \end{quote}
%
% L'installation de la classe consiste à créer le fichier
% |ulthese.cls| et plusieurs gabarits |.tex| à partir du code source
% documenté se trouvant dans le fichier |ulthese.dtx|. Il est
% recommandé de simplement créer ces fichiers dans le dossier de
% travail de la thèse ou du mémoire.
%
% Pour procéder à l'installation, décompresser l'archive |ulthese.zip|
% dans le dossier de travail, puis compiler avec {\LaTeX} le fichier
% |ulthese.ins| en exécutant
% \begin{quote}
%   |latex ulthese.ins|
% \end{quote}
% depuis une invite de commande. Si l'on est peu familier avec
% l'invite de commande, on peut aussi procéder comme avec tout
% document {\LaTeX}, soit ouvrir le fichier |ulthese.ins| dans son
% éditeur de texte favori et lancer depuis celui-ci la compilation avec {\LaTeX},
% pdf{\LaTeX}, {\XeLaTeX} ou un autre moteur {\TeX}.
%
% \section{Utilisation}
%
% La classe est compatible avec les moteurs {\LaTeX} traditionnels
% ainsi qu'avec le plus récent moteur {\XeLaTeX}.
%
% On charge la classe avec la commande
% \begin{quote}
%   |\documentclass[|\meta{options}|]{ulthese}|
% \end{quote}
% Les marges, l'interligne et la numérotation des pages sont adaptées
% aux règles de présentation matérielle de la FESP. Les options et les
% commandes définies par la classe sont décrites dans les sections
% suivantes.
%
% \subsection{Options de la classe}
% \label{sec:utilisation:options}
%
% Les \meta{options} que l'on peut spécifier au chargement de la classe
% sont les suivantes:
% \begin{description}
% \item[|PhD, MSc, MA, ...|] déclaration du type de grade (consulter
%   le \autoref{tab:grades} pour la liste complète des options et
%   les grades correspondants);
% \item[|multifacultaire|] déclaration d'une thèse multifacultaire;
% \item[|cotutelle|] déclaration d'une thèse effectuée en cotutelle;
% \item[|bidiplomation|] déclaration d'un mémoire en bidiplomation;
% \item[|extensionUdeS|] déclaration d'une thèse ou d'un mémoire
%   réalisé en extension à l'Université de Sherbrooke;
% \item[|extensionUQO|] déclaration d'une thèse ou d'un mémoire
%   réalisé en extension à l'Université du Québec en Outaouais;
% \item[|extensionUQAC|] déclaration d'une thèse ou d'un mémoire
%   réalisé en extension à l'Université du Québec à Chicoutimi;
% \item[|10pt|] sélectionne une taille de police de 10~points;
% \item[|11pt|] sélectionne une taille de police de 11~points;
% \item[|12pt|] sélectionne une taille de police de 12~points;
% \item[|nonatbib|] empêche le chargement du paquetage
%   \textsf{natbib};
% \item[|nobabel|] empêche le chargement du paquetage \textsf{babel};
% \item[|english, francais, ...|] langues utilisées dans le document;
% \item[\mdseries\meta{options memoir}] autres options du paquetage
%   \textsf{memoir}.
% \end{description}
%
% \begin{table}
%   \centering
%   \begin{tabular}{lp{9cm}}
%     \toprule
%     Option & Nom du grade (sigle) \\
%     \midrule
%     |LLD|\index{LLD=\verb+\LLD+}
%       & Docteur en droit (L.L.D.) \\
%     |DPsy|\index{DPsy=\verb+\DPsy+}
%       & Docteur en psychologie (D.Psy.) \\
%     |DThP|\index{DThP=\verb+\DThP+}
%       & Docteur en théologie pratique (D.Th.P.) \\
%     |PhD|\index{PhD=\verb+\PhD+}
%       & Philosophi{\ae} doctor (Ph.D.) \\
%     \addlinespace[6pt]
%     |LLM|\index{LLM=\verb+\LLM+}
%       & Maître en droit (L.L.M.) \\
%     |MA|\index{MA=\verb+\MA+}
%       & Maître ès arts (M.A.) \\
%     |MMus|\index{MMus=\verb+\MMus+}
%       & Maître en musique (M.Mus.) \\
%     |MSc|\index{MSc=\verb+\MSc+}
%       & Maître ès sciences (M.Sc.) \\
%     |MServSoc|\index{MServSoc=\verb+\MServSoc+}
%       & Maître en service social (M.Serv.Soc.) \\
%     |MScGeogr|\index{MScGeogr=\verb+\MScGeogr+}
%       & Maître en sciences géographiques (M.Sc.Géogr.) \\
%     |MATDR|\index{MATDR=\verb+\MATDR+}
%       & Maître en aménagement du territoire et développement régional
%       (M.ATDR) \\
%     \bottomrule
%   \end{tabular}
%   \caption{Options de la classe pour la déclaration du grade et libellés
%     correspondants}
%   \label{tab:grades}
% \end{table}
%
% La déclaration d'un type de grade est obligatoire.
%
% La déclaration d'une thèse multifacultaire nécessite d'utiliser la
% commande |\faculteUL|. La déclaration d'une thèse réalisée en
% cotutelle avec une autre université nécessite d'utiliser les
% commandes |\univcotutelle| et |\gradecotutelle|. De manière
% similaire, la déclaration d'un mémoire en bidiplomation avec une
% autre université nécessite d'utiliser les commandes
% |\univbidiplomation| et |\gradebidiplomation|. Les déclarations de
% thèse ou de mémoire réalisé en extension dans une autre université
% nécessitent d'utiliser les commandes |\faculteUL| et l'une ou
% l'autre de |\faculteUdeS|, |\faculteUQO| ou |\faculteUQAC|, selon le
% cas. L'ensemble de ces commandes sont décrites à la
% \autoref{sec:commandes}.
%
% Si aucune taille de police n'est précisée, la classe utilisera par
% défaut une police de 11~points. La taille des polices de la page
% titre n'est pas affectée par les options |10pt|, |11pt| et |12pt|.
%
% Le paquetage \textsf{natbib} est normalement chargé par la classe;
% voir la \autoref{sec:bibliographystyle}. L'option |nonatbib| permet
% d'empêcher le chargement pour modifier les options du paquetage ou
% en cas de conflit avec un autre paquetage de mise en forme de la
% bibliographie.
%
% La classe utilise par défaut le paquetage \textsf{babel} pour le
% traitement des langues dans le document; voir la
% \autoref{sec:langues}. L'option |nobabel| permet d'empêcher son
% chargement si un autre paquetage devait être utilisé --- on pense
% ici principalement à \textsf{polyglossia} pour un document produit
% avec le moteur {\XeLaTeX}.
%
% Les langues sont passées au paquetage \textsf{babel} (dans la mesure
% où |nobabel| n'est pas spécifié, bien entendu). Le libellé des
% langues devrait donc correspondre aux options de \textsf{babel}. La
% dernière langue spécifiée est la langue active par défaut dans le
% document.
%
% Toute autre option sera passée à la classe \textsf{memoir} dont,
% entre autres, le format du papier. Le format lettre nord-américain
% (option |letterpaper|) est utilisé par défaut. Si la thèse doit être
% imprimée en format international A4, utiliser l'option |a4paper|. La
% classe \textsf{memoir} est toujours chargée avec les options
% |twoside|, et |openright|.
%
% \subsection{Nouvelles commandes}
% \label{sec:commandes}
%
% La classe \textsf{ulthese} définit quelques nouvelles commandes
% servant principalement à créer la page titre et des éléments des pages
% liminaires.
%
% \subsubsection{Commandes de la page titre}
% \label{sec:commandes:pagetitre}
%
% Les commandes ci-dessous servent à définir les divers éléments de la
% page titre et leur disposition sur la page.
%
% \begin{DescribeMacro}{\titre}
%   Titre principal de la thèse ou du mémoire. Ne pas utiliser la
%   commande |\title| de {\LaTeX} pour ce faire.
%
%   Un titre très long devra être coupé manuellement avec |\\| ou
%   |\newline|. Par exemple, la déclaration d'un titre d'une seule
%   ligne est:
%   \begin{quote}
%     |\titre{Ceci est un titre d'une seule ligne}|
%   \end{quote}
%   Pour un titre de deux lignes, on écrira:
%   \begin{quote}
%     |\titre{Ceci est la première ligne d'un long titre \\| \\
%     |       et ceci est la seconde}|
%   \end{quote}
% \end{DescribeMacro}
%
% \begin{DescribeMacro}{\soustitre}
%   Sous-titre de la thèse ou du mémoire, le cas échéant. Les remarques
%   sur un long titre principal s'appliquent également au sous-titre.
% \end{DescribeMacro}
%
% \begin{DescribeMacro}{\auteur}
%   Nom complet de l'auteur de la thèse ou du mémoire, sous la forme
%   |Prénom Nom| avec seulement des majuscules initiales. Ne pas
%   utiliser la commande |\author| de {\LaTeX} pour le nom de l'auteur.
% \end{DescribeMacro}
%
% \begin{DescribeMacro}{\annee}
%   Année du dépôt final de la thèse ou du mémoire.
% \end{DescribeMacro}
%
% \begin{DescribeMacro}{\programme}
%   Nom complet officiel du programme d'études comme «Doctorat en
%   informatique» ou «Maîtrise en mathématiques». Si le programme
%   comporte une majeure, séparer sa mention de celle du programme
%   principal par un tiret demi-quadratin (obtenu avec |--|).
% \end{DescribeMacro}
%
% \begin{DescribeMacro}{\univcotutelle}
%   Prend effet seulement lorsque la classe est chargée avec
%   l'option |cotutelle|. Nom, ville et pays de l'université de
%   cotutelle, déclarés sous la forme
%   \begin{quote}
%     |\univcotutelle{Nom de l'université \\ Ville, Pays}|
%   \end{quote}
% \end{DescribeMacro}
%
% \begin{DescribeMacro}{\gradecotutelle}
%   Prend effet seulement lorsque la classe est chargée avec l'option
%   |cotutelle|. Grade conféré par l'université de cotutelle, déclaré
%   sous la forme
%   \begin{quote}
%     |\gradecotutelle{Nom du grade (sigle)}|
%   \end{quote}
% \end{DescribeMacro}
%
% \begin{DescribeMacro}{\univbidiplomation}
%   Prend effet seulement lorsque la classe est chargée avec l'option
%   |bidiplomation|. Nom, ville et pays de l'université de
%   bidiplomation, déclarés sous la forme
%   \begin{quote}
%     |\univbidiplomation{Nom de l'université \\ Ville, Pays}|
%   \end{quote}
% \end{DescribeMacro}
%
% \begin{DescribeMacro}{\gradebidiplomation}
%   Prend effet seulement lorsque la classe est chargée avec l'option
%   |bidiplomation|. Grade conféré par l'université de bidiplomation,
%   déclaré sous la forme
%   \begin{quote}
%     |\gradebidiplomation{Nom du grade (sigle)}|
%   \end{quote}
% \end{DescribeMacro}
%
% \begin{DescribeMacro}{\faculteUL}
%   Prend effet seulement lorsque la classe est chargée avec l'une ou
%   l'autre des options |multifacultaire|, |extensionUdeS|,
%   |extensionUQO| ou |extensionUQAC|. Cette macro a deux usages:
%   \begin{enumerate}
%   \item noms des facultés pour les thèses et mémoires
%     multifacultaires, séparés par des commandes |\\|;
%   \item nom de la faculté de l'Université Laval où sont réalisés les
%     thèses et mémoires en extension à l'Université de Sherbrooke, à
%     l'UQO ou à l'UQAC.
%   \end{enumerate}
% \end{DescribeMacro}
%
% \begin{DescribeMacro}{\faculteUdeS}
%   Prend effet seulement lorsque la classe est chargée avec l'option
%   |extensionUdeS|. Nom de la faculté de l'Université de Sherbrooke
%   hébergeant la thèse en extension.
% \end{DescribeMacro}
%
% \begin{DescribeMacro}{\faculteUQO}
%   Prend effet seulement lorsque la classe est chargée avec l'option
%   |extensionUQO|. Nom de la faculté de l'Université du Québec en
%   Outaouais hébergeant la thèse en extension.
% \end{DescribeMacro}
%
% \begin{DescribeMacro}{\faculteUQAC}
%   Prend effet seulement lorsque la classe est chargée avec l'option
%   |extensionUQAC|. Nom de la faculté de l'Université du Québec à
%   Chicoutimi hébergeant le mémoire en extension.
% \end{DescribeMacro}
%
% \begin{DescribeMacro}{\pagetitre}
%   Déclaration de création de la page titre. Ne pas utiliser la
%   commande |\pagetitle| de {\LaTeX} pour ce faire. De toutes les
%   commandes ci-dessus, c'est la seule qui doit se trouver dans le
%   corps du document plutôt que dans le préambule.
% \end{DescribeMacro}
%
% \subsubsection{Commandes des pages liminaires}
%
% La classe définit deux commandes pour créer des pages liminaires
% prévues aux règles de présentation matérielle.
%
% \begin{DescribeMacro}{\dedicace}
%   La commande |\dedicace| ajoute une dédicace («À mes parents», «À
%   Camille») à la thèse ou au mémoire. La dédicace est disposée seule
%   sur une page recto, à une dizaine de lignes de la marge du haut et
%   alignée à droite. Par défaut, elle est composée en italique.
% \end{DescribeMacro}
%
% \begin{DescribeMacro}{\epigraphe}
%   La commande |\epigraph| sert à ajouter une épigraphe au début du
%   document. Comme la dédicace, l'épigraphe est disposée seule sur une
%   page recto, à une dizaine de lignes de la marge du haut et alignée à
%   droite. La commande accepte deux arguments, soit le texte de la
%   citation et son auteur ou la source, dans l'ordre.
% \end{DescribeMacro}
%
% Pour ajouter une épigraphe au début d'un ou plusieurs chapitres,
% utiliser directement la commande |\epigraph| de \textsf{memoir}, sur
% laquelle |\dedicace| et |\epigraphe| sont d'ailleurs basées.
%
% \subsection{Citations}
% \label{sec:citations}
%
% {\LaTeX} offre deux environnements pour les citations dans le
% texte: |quote| et |quotation|.
%
% \begin{DescribeEnv}{quote}
%   L'environnement |quote| sert pour les citations «courtes»,
%   quelques lignes au plus. Dans la classe, le texte est alors placé
%   en retrait des marges normales de 10~mm à gauche et à droite.
% \end{DescribeEnv}
%
% \begin{DescribeEnv}{quotation}
%   L'environnement |quotation|, quant à lui, doit être utilisé pour
%   les citations «longues», celles qui peuvent s'étendre sur plus de
%   cinq lignes ou, surtout, plus d'un paragraphe. Dans la classe, le
%   texte est alors toujours placé en retrait de 10~mm, mais également
%   à interligne simple. De plus, les paragraphes, le cas échéant,
%   sont séparés d'un espace vertical afin de bien les distinguer les
%   uns des autres.
% \end{DescribeEnv}
%
% \subsection{Interligne}
%
% \begin{DescribeMacro}{\OnehalfSpacing}
%   L'espacement d'un interligne et demi utilisé dans la classe est
%   obtenu avec la commande |\OnehalfSpacing| de \textsf{memoir}.
%   L'interligne simple est automatiquement rétabli pour la page
%   titre, la table des matières, la liste des tableaux, la liste
%   des figures et les longues citations (\autoref{sec:citations}).
% \end{DescribeMacro}
%
% \begin{DescribeMacro}{\SingleSpacing}
%   Si ce devait être nécessaire ailleurs dans le document, la
%   commande |\SingleSpacing| permet de passer à l'interligne simple.
% \end{DescribeMacro}
%
%
% \subsection{Autres paquetages chargés}
% \label{sec:paquetages}
%
% Outre \textsf{memoir}, la classe \textsf{ulthese} charge quelques
% paquetages qui peuvent aussi être utiles pour l'utilisateur de la
% classe. Il n'est donc pas nécessaire de charger de nouveau les
% paquetages suivants:
% \begin{description}
% \item[\normalfont\textsf{babel}] gestion des documents rédigés dans
%   une ou plusieurs langues autres que l'anglais (si l'option
%   |nobabel| de la classe est absente; voir aussi la
%   \autoref{sec:langues});
% \item[\normalfont\textsf{numprint}] requis par la commande |\nombre|
%   de \textsf{babel}; le paquetage est donc chargé uniquement si
%   \textsf{babel} l'est. Permet de composer automatiquement des
%   nombres avec un séparateur toutes les trois positions (une espace
%   en français);
% \item[\normalfont\textsf{natbib}] gestion de la bibliographie (si
%   l'option |nonatbib| de la classe est absente; voir aussi la
%   \autoref{sec:bibliographystyle});
% \item[\normalfont\textsf{fontspec}] gestion des polices OpenType
%   sous {\XeLaTeX} (chargé avec ce moteur seulement);
% \item[\normalfont\textsf{graphicx}] support pour l'insertion et la
%   manipulation de graphiques;
% \item[\normalfont\textsf{xcolor}] extension du paquetage
%   \textsf{color} pour gérer les couleurs dans le texte;
% \item[\normalfont\textsf{textcomp}] multitude de symboles spéciaux,
%   dont un beau symbole de copyright, \textcopyright.
% \end{description}
% L'\autoref{sec:meo} sur la mise en {\oe}uvre de la classe fournit
% plus de détails sur la liste des paquetages chargés et les raisons
% pour lesquelles ils sont requis dans la classe.
%
% \section{Français et autres langues}
% \label{sec:langues}
%
% Une complication additionnelle pour les auteurs rédigeant dans une
% langue autre que l'anglais consiste à adapter {\LaTeX} à leur
% langue, qu'il s'agisse des mots clés, de la typographie ou de la
% césure des mots. La solution standard à ce problème provient du
% paquetage \textsf{babel}. Celui-ci permet de combiner plusieurs
% langues dans un même document et de passer de l'une à l'autre
% facilement. Il est chargé par défaut par la classe \textsf{ulthese}.
%
% Aucune langue n'est spécifiée dans la classe. La plupart des auteurs
% auront recours à l'anglais et au français, ne serait-ce que pour les
% deux résumés demandés par la FESP. Les langues utilisées dans le
% document doivent être spécifiées comme options à la classe, tel que
% mentionné à la \autoref{sec:utilisation:options}. La
% \emph{dernière} langue spécifiée devient par défaut la langue active
% du document.
%
% \begin{DescribeMacro}{\selectlanguage}
%   La commande |\selectlanguage| de \textsf{babel} permet de passer de
%   la langue courante à la langue spécifiée en argument.
% \end{DescribeMacro}
%
% \begin{DescribeEnv}{otherlanguage}
%   L'environnement |otherlanguage| de \textsf{babel} permet de faire
%   la même chose que la commande |\selectlanguage|, sauf que le
%   changement de langue est local à l'environnement --- utile pour
%   les brefs changements de langue.
% \end{DescribeEnv}
%
% Si vous n'êtes pas autrement familier avec le paquetage
% \textsf{babel}, consulter sa documentation. Celle-ci est éclatée en un
% document principal, |babel.pdf|, pour le c{\oe}ur du paquetage et plusieurs
% autres pour les fonctionnalités propres à une langue (|english.pdf|,
% |frenchb.pdf|, etc.). Consulter au moins les documents consacrés aux
% langues utilisées dans votre thèse ou mémoire. Le plus simple
% consiste sans doute à consulter en ligne sur CTAN la  %
% \href{http://mirrors.ctan.org/macros/latex/required/babel/base/babel.pdf}{%
% documentation de base} %
% et les
% \href{http://mirrors.ctan.org/macros/latex/required/babel/contrib/}{%
% documents spécifiques par langue}.
%
% Les utilisateurs de {\XeLaTeX} qui souhaiteraient plutôt utiliser le
% plus récent paquetage \textsf{polyglossia} peuvent empêcher le
% chargement de \textsf{babel} avec l'option |nobabel| de la classe.
% Ils devront toutefois charger et configurer \textsf{polyglossia}
% eux-mêmes dans l'entête de leur document. Ce paquetage est moins
% évolué que \textsf{babel} pour la typographie française.
%
% \section{Police de caractères du document}
% \label{sec:police}
%
% Les documents {\LaTeX} sont facilement reconnaissables par leur
% police de caractères par défaut, {\fontfamily{cmr}\selectfont
% Computer Modern}. Avec toute distribution {\LaTeX} moderne, il est
% maintenant simple d'utiliser l'une ou l'autre des polices Postscript
% standards. D'ailleurs la classe \textsf{ulthese} utilise la police
% sans empattements \textsf{Helvetica} pour composer la page titre.
%
% La FESP permet l'utilisation des polices
% {\fontfamily{ptm}\selectfont Times} et Palatino\footnote{%
% Palatino est la police utilisée dans le présent document.} %
% dans les thèse et mémoires {\LaTeX}. Pour utiliser ces polices avec
% {\LaTeX}, charger les paquetages \textsf{mathptmx} ou
% \textsf{mathpazo}, respectivement. Pour les détails, consulter la
% documentation de l'ensemble de paquetages PSNFSS. Rechercher sur
% votre système le fichier |psnfss2e.pdf| ou le %
% \href{http://mirrors.ctan.org/macros/latex/required/psnfss/psnfss2e.pdf}{%
% consulter en ligne} %
% sur CTAN.
%
% Avec {\XeLaTeX}, on peut utiliser les polices
% {\fontfamily{qtm}\selectfont Termes} et {\fontfamily{qpl}\selectfont
% Pagella} du projet
% \href{http://www.gust.org.pl/projects/e-foundry/tex-gyre/}{TeX
% Gyre}. Ce sont des polices très similaires à Times et Palatino,
% disponibles en version OpenType et qui fournissent un bon support
% pour les mathématiques via le projet frère
% \href{http://www.gust.org.pl/projects/e-foundry/tg-math/}{TeX Gyre
% Math}. La gestion des polices de caractères avec {\XeLaTeX} se fait
% avec le paquetage standard \textsf{fontspec} --- documentation dans le
% fichier |fontspec.pdf| ou %
% \href{http://mirrors.ctan.org/macros/latex/contrib/fontspec/fontspec.pdf}{%
% directement sur CTAN}.
%
% \section{Gabarits}
%
% Il est recommandé de segmenter tout document d'une certaine ampleur
% dans des fichiers |.tex| distincts pour chaque partie ---
% habituellement un fichier par chapitre. Le document complet est
% composé à l'aide d'un fichier maître qui contient le préambule
% {\LaTeX} et un ensemble de commandes
% |\include|\index{include=\verb+\include+} pour réunir les parties
% dans un tout.
%
% La classe \textsf{ulthese} est livrée avec un ensemble de gabarits sur
% lesquels se baser pour:
% \begin{itemize}
% \item les fichiers maîtres de divers types de thèses et mémoires
%   (standard, sur mesure, en cotutelle, en bidiplomation, en
%   extension, etc.);
% \item les fichiers des parties les plus usuelles (résumés français
%   et anglais, avant-propos, introduction, chapitres, conclusion,
%   etc.).
% \end{itemize}
% Les noms des fichiers devraient permettre de facilement identifier
% leur contenu (une bonne pratique; |rappels.tex| est plus parlant et
% résiste mieux aux changements à l'ordre des chapitres que
% |chapitre1.tex|).
%
% Pour débuter la rédaction, renommer le gabarit de document maître
% approprié d'après votre numéro de dossier. Par exemple, l'étudiante
% dont le numéro de dossier est 900352789 et qui entame la rédaction
% d'une thèse multifacultaire renommera le fichier
% \begin{quote}
%   |gabarit-doctorat-multifacultaire.tex|
% \end{quote}
% en
% \begin{quote}
%   |900352789.tex|.
% \end{quote}
% Les autres gabarits de documents maîtres peuvent alors être
% supprimés.
%
% Les gabarits comportent des commentaires succincts pour vous guider
% dans la préparation de votre document. Les sections suivantes
% fournissent des détails additionnels, et ce, dans l'ordre où les
% commandes apparaissent dans les gabarits.
%
% \subsection{Encodage des fichiers}
%
% Taper de longs textes en français en {\LaTeX} devient rapidement
% pénible si on utilise les commandes |\'e|, |\`a| ou |\^e| pour entrer
% les lettres accentuées. Afin de pouvoir plutôt entrer directement |é|,
% |à| ou |ê|, {\LaTeX} doit être configuré pour reconnaître les lettres
% accentuées. C'est le rôle du paquetage \textsf{inputenc}.
%
% Cependant, il existe plusieurs manières différentes d'encoder --- ou
% d'enregistrer --- les lettres accentuées et autres caractères
% spéciaux (comme, par exemple, le symbole de l'euro) dans un
% ordinateur. La méthode la plus répandue et celle standard sur les
% versions récentes des systèmes d'exploitation Linux et OS~X est
% l'UTF-8 de la norme
% \href{http://fr.wikipedia.org/wiki/Unicode}{Unicode}. Les gabarits
% sont livrés dans ce type d'encodage.
%
% La déclaration
% \begin{quote}
%   |\usepackage[utf8]{inputenc}|
% \end{quote}
% dans le préambule assure que {\LaTeX} traitera correctement des
% fichiers source encodés en UTF-8.
%
% La norme Unicode n'est pas aussi uniformément supportée par Windows.
% Selon l'éditeur de texte employé et la version du système
% d'exploitation, il peut être nécessaire d'utiliser les normes
% d'encodage %
% \href{http://fr.wikipedia.org/wiki/ISO_8859-1}{ISO~8859-1} %
% (ou Latin-1; option |latin1| de \textsf{inputenc}), %
% \href{http://fr.wikipedia.org/wiki/ISO_8859-15}{ISO~8859-15} %
% (ou Latin-9; option |latin9|) ou %
% \href{http://fr.wikipedia.org/wiki/Windows-1252}{Windows-1252} %
% (options |cp1252| ou |ansinew|).
%
% La situation est plus simple avec le moteur {\XeLaTeX} puisqu'il
% gère nativement Unicode. Le paquetage \textsf{inputenc} est non
% seulement inutile, mais incompatible avec {\XeLaTeX}. C'est
% pourquoi, dans les gabarits, \textsf{inputenc} est chargé seulement
% lorsque {\XeLaTeX} n'est pas le moteur employé pour compiler le
% document.
%
% \subsection{Paquetages additionnels}
%
% Tel qu'explicité à la \autoref{sec:paquetages}, la classe
% charge déjà quelques paquetages. Cependant, il est fort probable que
% vous devrez en charger d'autres pour composer votre document. Les
% gabarits prévoient un endroit pour le chargement de paquetages
% additionnels. Il est recommandé d'inscrire vos commandes
% |\usepackage| à cet endroit afin de respecter un certain ordre de
% chargement; voir ci-dessous.
%
% Si vous utilisez un paquetage non standard dans les distributions
% courantes (MiK\TeX, \TeX~Live, Mac\TeX), vous devez le fournir avec
% le code source de votre document lors du dépôt final.
%
% \subsection{Changement de police de caractères}
%
% Les gabarits comportent des déclarations types pour utiliser les
% polices Palatino ou Times sous {\LaTeX} ou, sous {\XeLaTeX}, leurs
% équivalents Pagella et Termes du projet TeX~Gyre.
%
% \subsection{Hyperliens}
% \label{sec:hyperref}
%
% Le paquetage \textsf{hyperref} permet de transformer toutes les
% références en hyperliens cliquables lorsque le document est produit
% avec pdf{\LaTeX}. L'interaction de ce paquetage avec les autres est
% parfois (voire souvent) délicate. Pour cette raison, il est
% habituellement nécessaire de charger \textsf{hyperref} en tout
% dernier. C'est pourquoi il n'est pas chargé dans la classe, mais
% plutôt dans les gabarits. Prendre soin de maintenir le dernier rang
% de chargement lors de l'édition d'un gabarit.
%
% La configuration du paquetage dans les gabarits fait en sorte que
% les liens sont simplement signalés par une couleur de texte
% légèrement contrastante. L'utilisation de couleurs dans un document
% requiert le paquetage |xcolor|, chargé par la classe. La couleur de
% lien par défaut, |ULlinkcolor|, est définie dans la classe; voir la
% \autoref{sec:couleurs}.
%
% \subsection{Options de \textsf{babel}}
%
% \begin{DescribeMacro}{\frenchbsetup}
%   La commande |\frenchbsetup|\index{frenchbsetup=\verb+\frenchbsetup+}
%   de \textsf{babel} permet de contrôler certains ajustements
%   typographiques apportés par le paquetage en mode français. Consulter
%   la documentation de \textsf{babel} pour la liste des options de
%   configuration disponibles.
% \end{DescribeMacro}
%
% Les concepteurs de la classe \textsf{ulthese}  proposent deux
% ajustements dans les gabarits:
% \begin{enumerate}
% \item l'option |CompactItemize=false| évite que le mode français de
%   \textsf{babel} ne diminue l'espacement vertical dans les listes;
% \item avec l'option |ThinSpaceInFrenchNumbers=true|, une espace fine
%   sera utilisée comme séparateur des milliers dans les nombres plutôt
%   qu'une espace pleine.
% \end{enumerate}
% \begin{DescribeMacro}{\nombre}
%   D'ailleurs, à ce sujet, le paquetage \textsf{numprint} étant
%   chargé dans la classe avec \textsf{babel}, on peut utiliser la
%   commande |\nombre| pour formater automatiquement les nombres. Par
%   exemple, le résultat de |\nombre{123456789}| est
%   \nombre{123456789}.
% \end{DescribeMacro}
%
% Ces ajustements doivent évidemment être désactivés si l'option
% |nobabel| est spécifiée au chargement de la classe.
%
% \subsection{Style de la bibliographie}
% \label{sec:bibliographystyle}
%
% \begin{DescribeMacro}{\bibliographystyle}
%   Il est fortement recommandé d'utiliser BIB{\TeX} pour la préparation
%   de la bibliographie. Le formatage de la bibliographie est contrôlé
%   par un style choisi par la commande |\bibliographystyle|. Les styles
%   standards de {\LaTeX} sont |plain|, |unsrt|, |alpha| et |abbrv|.
% \end{DescribeMacro}
%
% Pour plus de flexibilité, il est recommandé d'utiliser le paquetage
% \textsf{natbib} pour la gestion des références et des styles de la
% bibliographie. Entre autres choses, ce paquetage supporte le style de
% citation auteur-année fréquemment employé en sciences naturelles,
% plusieurs commandes de citation, un grand nombre de styles de
% bibliographie ainsi que des entrées spécifiques pour les numéros ISBN
% et les URL. Le paquetage fournit des styles de bibliographie
% |plainnat|, |unsrtnat| et |abbrvnat| similaires aux styles standards,
% mais plus complets. Il existe des
% \href{http://mirrors.ctan.org/biblio/bibtex/contrib/bib-fr/}{versions
%   francisées} de ces styles (et de quelques autres) dans CTAN.
%
% Afin d'assurer le bon fonctionnement avec \textsf{babel}, le
% paquetage \textsf{natbib} est chargé par la classe \textsf{ulthese}
% (à moins que l'option |nonatbib| ne soit spécifiée) avec les options
% par défaut, soit |round|, |semicolon| et |authoryear|. Pour
% spécifier d'autres options, vous avez deux possibilités:
% \begin{enumerate}
% \item utiliser l'option |nonatbib| de la classe et ensuite charger
%   explicitement \textsf{natbib} avec ses options;
% \item
%   \begin{DescribeMacro}{\setcitestyle}
%     passer de nouvelles options à \textsf{natbib} avec la
%     commande |\setcitestyle|.
%   \end{DescribeMacro}
% \end{enumerate}
%
% Par exemple, pour utiliser un style de citation numérique où le
% numéro de la référence se trouve entre crochets, on peut procéder de
% l'une de ces deux manières:
% \begin{quote}
%   |\documentclass[nonatbib]{ulthese}| \\
%   |\usepackage[numbers,square]{natbib}| \\
%   |...| \\
%   |\bibliographystyle{plain-fr}|
% \end{quote}
% ou
% \begin{quote}
%   |\documentclass{ulthese}| \\
%   |...| \\
%   |\bibliographystyle{plain-fr}| \\
%   |\setcitestyle{numbers,square}|
% \end{quote}
%
% La documentation de \textsf{natbib} se trouve dans le fichier
% |natbib.pdf| sur votre système ou %
% \href{http://mirrors.ctan.org/macros/latex/contrib/natbib/natbib.pdf}{%
% en ligne} %
% sur CTAN.
%
% On trouve dans CTAN le paquetage
% \href{http://www.ctan.org/pkg/francais-bst/}{\textsf{francais-bst}}
% qui fournit une feuille de style compatible avec \textsf{natbib}.
% Celle-ci permet de composer des bibliographies auteur-année
% respectant les normes de typographie française proposées dans
% \cite{Malo:1996}. Pour utiliser ce style, on spécifiera dans le
% préambule du document LaTeX
% \begin{quote}
%   |\bibliographystyle{francais}|
% \end{quote}
%
% Autrement, la FESP n'a pas d'exigences particulières quant à la
% présentation de la bibliographie (présentation du titre, des auteurs
% et autres informations bibliographiques).
%
% \subsection{Déclarations de la page titre}
%
% Les gabarits comportent toutes les déclarations nécessaires pour
% composer la page titre des divers types de thèse ou de mémoires.
% Vous devez remplacer les éléments se trouvant entre crochets <~> en
% respectant la forme indiquée. Assurez-vous de supprimer les
% caractères < et > afin qu'ils n'apparaissent pas sur la page titre de
% votre document.
%
% \subsection{Pages liminaires}
%
% \begin{DescribeMacro}{\frontmatter}
%   La commande |\frontmatter| déclare que {\LaTeX} doit considérer le
%   matériel qui suit comme des pages liminaires. En pratique, cela
%   résulte essentiellement en une numérotation des pages en chiffres
%   romains.
% \end{DescribeMacro}
%
% Les normes de présentation de la FESP édictent que les thèses et
% mémoires devraient comporter les pages liminaires suivantes, dans
% l'ordre:
% \begin{enumerate}
% \item la page titre (obligatoire);
% \item un résumé en français (obligatoire);
% \item un résumé en anglais (recommandé mais non obligatoire);
% \item une table des matières (obligatoire);
% \item une liste des tableaux;
% \item une liste des figures;
% \item une liste des abbréviations et des sigles;
% \item une dédicace;
% \item une épigraphe;
% \item des remerciements;
% \item un avant-propos (obligatoire dans le cas d'un mémoire ou d'une
%   thèse avec insertion d'articles).
% \end{enumerate}
%
% Les commandes
% \begin{quote}
%   |\pagetitre| \\
%   |\tableofcontents| \\
%   |\listoftables| \\
%   |\listoffigures| \\
%   |\dedicace{|\meta{texte}|}| \\
%   |\epigraphe{|\meta{texte}|}{|\meta{auteur}|}|
% \end{quote}
% permettent de générer les pages correspondantes. Seules les deux
% dernières commandes admettent des arguments.
%
% \begin{DescribeMacro}{\chapter*}
%   Les résumés, la liste des abbréviations et des sigles, les
%   remerciements et l'avant-propos sont composés comme des chapitres
%   normaux, mais sans être numérotés. Il faut donc définir ces éléments
%   avec la commande |\chapter*|.
% \end{DescribeMacro}
%
% \begin{DescribeMacro}{\phantomsection}
%   \begin{DescribeMacro}{\addcontentsline}
%     Les sections declarées avec la commande |\chapter*|
%     n'apparaissent pas dans la table des matières. Comme les normes
%     de présentation de la FESP exigent que toutes les pages
%     liminaires y figurent, on fait suivre les commandes
%     |\chapter*{|\meta{Titre}|}| des commandes
%   \end{DescribeMacro}
% \end{DescribeMacro}
% \begin{quote}
%   |\phantomsection\addcontentsline{toc}{chapter}{|\meta{Titre}|}|
% \end{quote}
% Celles-ci ajoutent à la table des matières (|toc|) une section de
% niveau |chapter| dont le titre est \meta{Titre}. La commande
% |\phantomsection| est rendue nécessaire (ou recommandée) par le
% paquetage \textsf{hyperref}.
%
% \subsection{Corps du document}
%
% \begin{DescribeMacro}{\mainmatter}
%   La commande |\mainmatter| délimite le début du corps du document. La
%   numérotation des pages passe en chiffres arabes.
% \end{DescribeMacro}
%
% Le corps du document devrait normalement compter une introduction (non
% numérotée), un développement divisé en chapitres (numérotés) et une
% conclusion (non numérotée).
%
% \subsection{Annexes}
%
% \begin{DescribeMacro}{\appendix}
%   Si la thèse ou le mémoire comporte une ou plusieurs annexes,
%   composer celles-ci comme des chapitres normaux insérés dans le
%   document maître après la commande |\appendix|. Cette commande a
%   pour effet de passer d'un mode de numération numérique (1, 2, 3,
%   \dots) à un mode alphabétique (A, B, C, \dots).
% \end{DescribeMacro}
%
% \subsection{Bibliographie}
%
% \begin{DescribeMacro}{\bibliography}
%   Si vous utilisez BIB{\TeX}, la bibliographie est insérée dans le
%   document à l'endroit où apparait la commande |\bibliography| dans le
%   code source.
% \end{DescribeMacro}
%
% Consulter la \autoref{sec:bibliographystyle} pour des
% informations additionnelles sur la préparation de la bibliographie.
%
% \section{Aide additionnelle}
%
% Pour obtenir de l'aide additionnelle sur l'utilisation de la classe
% \textsf{ulthese} (et non sur celle de {\LaTeX} en général), prière
% de consulter d'abord
% \begin{enumerate}
% \item le \href{http://www.theses.ulaval.ca/wiki/}{WikiThèse} de
%   l'Université Laval, en particulier la
%   \href{http://www.theses.ulaval.ca/wiki/index.php?title=FAQ}{Foire aux questions};
% \item les
%   \href{http://listes.ulaval.ca/listserv/archives/ulthese-aide.html}{archives}
%   de la liste de distribution \texttt{ulthese-aide}.
% \end{enumerate}
% Si la réponse à votre question ne se trouve ni dans le wiki, ni dans
% les archives, alors écrire à l'adresse
% \href{mailto:ulthese-aide@listes.ulaval.ca}{\url{ulthese-aide@listes.ulaval.ca}}.
%
% \StopEventually{\PrintChanges\PrintIndex}
%
% ^^A Début du code de la classe
%
% \newpage
% \appendix
% \section{Mise en {\oe}uvre}
% \label{sec:meo}
%
% Cette annexe passe en revue le code {\TeX} et {\LaTeX} de la
% classe. Elle ne risque d'intéresser que les personnes qui souhaitent
% explorer comment la classe est programmée.
%
% \subsection{Tests}
%
% Les paquetages \textsf{ifthen} et \textsf{ifxetex} sont nécessaires
% pour effectuer divers tests dans la classe.
%    \begin{macrocode}
%<*class>
\RequirePackage{ifthen}
\RequirePackage{ifxetex}
%    \end{macrocode}
%
% \subsection{Options de la classe}
%
% Il y a quatre grandes catégories d'options propres à la classe: la
% possibilité d'empêcher le chargement du paquetage \textsf{natbib};
% la possibilité d'empêcher le chargement du paquetage \textsf{babel};
% la taille de la police de caractères en points; le type de grade.
%
% \begin{macro}{nonatbib}
%   L'option |nonatbib| permet d'empêcher la classe de charger le
%   paquetage \textsf{natbib} en cas d'incompatibilité avec d'autres
%   paquetages spécialisés de mise en forme de la bibliographie.
%    \begin{macrocode}
\newboolean{UL@natbib}
\setboolean{UL@natbib}{true}
\DeclareOption{nonatbib}{\setboolean{UL@natbib}{false}}
%    \end{macrocode}
% \end{macro}
%
% \begin{macro}{nobabel}
%   L'option |nobabel| permet d'empêcher la classe de charger le
%   paquetage \textsf{babel}. Cette option peut s'avérer utile pour
%   les utilisateurs de {\XeLaTeX} qui souhaitent plutôt utiliser
%   \textsf{poyglossia} pour le traitement des langues dans leur
%   document.
%    \begin{macrocode}
\newboolean{UL@babel}
\setboolean{UL@babel}{true}
\DeclareOption{nobabel}{\setboolean{UL@babel}{false}}
%    \end{macrocode}
% \end{macro}
%
% \begin{macro}{10pt,11pt,12pt}
%   Les valeurs possibles pour la taille de la police de caractères
%   sont |10pt|, |11pt| et |12pt|. Cette option est gérée au niveau de
%   la classe afin de s'assurer que les divers éléments sur la page
%   titre sont toujours de la même taille. La taille de la police par
%   défaut permet de déterminer si, par exemple, le titre du document
%   doit être dans la taille |\Huge|, |\huge| ou |\LARGE| de
%   \textsf{memoir}.
%
%   La taille de la police est passée à \textsf{memoir} et la macro
%   |\UL@ptsize| stocke la taille des caractères pour usage futur.
%    \begin{macrocode}
\newcommand*{\UL@ptsize}{}
\DeclareOption{10pt}{%
  \PassOptionsToClass{10pt}{memoir}
  \renewcommand*{\UL@ptsize}{10}}
\DeclareOption{11pt}{%
  \PassOptionsToClass{11pt}{memoir}
  \renewcommand*{\UL@ptsize}{11}}
\DeclareOption{12pt}{%
  \PassOptionsToClass{12pt}{memoir}
  \renewcommand*{\UL@ptsize}{12}}
%    \end{macrocode}
% \end{macro}
%
% \begin{macro}{PhD,MSc,MA,...}
%   Définition du type de grade et si la thèse ou le mémoire est
%   multifacultaire, effectué en cotutelle, en bidiplomation ou en
%   extension. On définit également une valeur booléenne pour stocker
%   si le type de programme (doctorat ou maîtrise) est masculin ou
%   non; cela servira à adapter la composition de la page titre, plus
%   loin.
%    \begin{macrocode}
\newboolean{UL@isprogmasc}
\newcommand*{\UL@typenum}{}
\DeclareOption{LLD}{%
  \renewcommand*{\UL@typenum}{0}
  \setboolean{UL@isprogmasc}{true}
  \newcommand*{\UL@typeofdoc}{Th\`ese}
  \newcommand*{\UL@degree}{Docteur en droit (L.L.D.)}}
\DeclareOption{DPsy}{%
  \renewcommand*{\UL@typenum}{0}
  \setboolean{UL@isprogmasc}{true}
  \newcommand*{\UL@typeofdoc}{Th\`ese}
  \newcommand*{\UL@degree}{Docteur en psychologie (D.Psy.)}}
\DeclareOption{DThP}{%
  \renewcommand*{\UL@typenum}{0}
  \setboolean{UL@isprogmasc}{true}
  \newcommand*{\UL@typeofdoc}{Th\`ese}
  \newcommand*{\UL@degree}{Docteur en th\'eologie pratique (D.Th.P.)}}
\DeclareOption{PhD}{%
  \renewcommand*{\UL@typenum}{0}
  \setboolean{UL@isprogmasc}{true}
  \newcommand*{\UL@typeofdoc}{Th\`ese}
  \newcommand*{\UL@degree}{Philosophi{\ae} doctor (Ph.D.)}}
\DeclareOption{LLM}{%
  \renewcommand*{\UL@typenum}{0}
  \setboolean{UL@isprogmasc}{false}
  \newcommand*{\UL@typeofdoc}{M\'emoire}
  \newcommand*{\UL@degree}{Ma\^itre en droit (L.L.M.)}}
\DeclareOption{MA}{%
  \renewcommand*{\UL@typenum}{0}
  \setboolean{UL@isprogmasc}{false}
  \newcommand*{\UL@typeofdoc}{M\'emoire}
  \newcommand*{\UL@degree}{Ma\^itre \`es arts (M.A.)}}
\DeclareOption{MMus}{%
  \renewcommand*{\UL@typenum}{0}
  \setboolean{UL@isprogmasc}{false}
  \newcommand*{\UL@typeofdoc}{M\'emoire}
  \newcommand*{\UL@degree}{Ma\^itre en musique (M.Mus.)}}
\DeclareOption{MSc}{%
  \renewcommand*{\UL@typenum}{0}
  \setboolean{UL@isprogmasc}{false}
  \newcommand*{\UL@typeofdoc}{M\'emoire}
  \newcommand*{\UL@degree}{Ma\^itre \`es sciences (M.Sc.)}}
\DeclareOption{MServSoc}{%
  \renewcommand*{\UL@typenum}{0}
  \setboolean{UL@isprogmasc}{false}
  \newcommand*{\UL@typeofdoc}{M\'emoire}
  \newcommand*{\UL@degree}{Ma\^itre en service social (M.Serv.Soc.)}}
\DeclareOption{MScGeogr}{%
  \renewcommand*{\UL@typenum}{0}
  \setboolean{UL@isprogmasc}{false}
  \newcommand*{\UL@typeofdoc}{M\'emoire}
  \newcommand*{\UL@degree}{Ma\^itre en sciences g\'eographiques (M.Sc.G\'eogr.)}}
\DeclareOption{MATDR}{%
  \renewcommand*{\UL@typenum}{0}
  \setboolean{UL@isprogmasc}{false}
  \newcommand*{\UL@typeofdoc}{M\'emoire}
  \newcommand*{\UL@degree}{Ma\^itre en am\'enagement du territoire et d\'eveloppement r\'egional (M.ATDR)}}
\DeclareOption{multifacultaire}{%
  \renewcommand*{\UL@typenum}{1}}
\DeclareOption{cotutelle}{%
  \renewcommand*{\UL@typenum}{2}
  \protected@edef\UL@typeofdoc{\UL@typeofdoc\ en cotutelle}}
\DeclareOption{bidiplomation}{%
  \renewcommand*{\UL@typenum}{2}
  \protected@edef\UL@typeofdoc{\UL@typeofdoc}}
\DeclareOption{extensionUdeS}{%
  \renewcommand*{\UL@typenum}{3}
  \newcommand*{\UL@extensionat}{Universit\'e de Sherbrooke}
  \newcommand*{\UL@extensionloc}{Sherbrooke, Qu\'ebec}}
\DeclareOption{extensionUQO}{%
  \renewcommand*{\UL@typenum}{3}
  \newcommand*{\UL@extensionat}{Universit\'e du Qu\'ebec en Outaouais}
  \newcommand*{\UL@extensionloc}{Gatineau, Qu\'ebec}}
\DeclareOption{extensionUQAC}{%
  \renewcommand*{\UL@typenum}{3}
  \newcommand*{\UL@extensionat}{Universit\'e du Qu\'ebec \`a Chicoutimi}
  \newcommand*{\UL@extensionloc}{Chicoutimi, Qu\'ebec}}
%    \end{macrocode}
% \end{macro}
%
% \subsection{Chargement de la classe \textsf{memoir}}
%
% Toutes les options de la classe sont passées à \textsf{memoir}. Le
% format de papier et la taille de police par défaut sont, dans
% l'ordre, |letterpaper| et |11pt|. On vérifie qu'un type de grade a
% bien été déclaré. Les options de \textsf{memoir} |twoside| et
% |openright| sont explicitement déclarées afin d'éviter toute
% tentative de passer outre à ces exigences de la FESP.
%    \begin{macrocode}
\DeclareOption*{\PassOptionsToClass{\CurrentOption}{memoir}}
\ExecuteOptions{11pt,letterpaper}
\ProcessOptions
\ifx\UL@typenum\empty
  \ClassError{ulthese}{%
    No thesis type specified.}
    {Declare the thesis type as a class option.}
\fi
\LoadClass[twoside,openright]{memoir}
%    \end{macrocode}
%
% \subsection{Paquetages requis}
%
% La classe s'efforce de charger un minimum de paquetages afin d'éviter
% les conflits potentiels.
%
% {\XeLaTeX} requiert le paquetage \textsf{fontspec} pour le
% traitement des polices. Le paquetage \textsf{unicode-math} facilite
% également le traitement des polices et des symboles mathématiques
% avec ce moteur. Sous {\LaTeX}, il est aujourd'hui préférable
% d'utiliser les polices T1.
%    \begin{macrocode}
\ifxetex
  \RequirePackage{fontspec}
  \RequirePackage{unicode-math}
  \defaultfontfeatures{Ligatures=TeX}
\else
  \RequirePackage[T1]{fontenc}
\fi
%    \end{macrocode}
%
% Le paquetage \textsf{natbib} doit être chargé avant \textsf{babel}
% pour bien fonctionner. C'est pourquoi il est chargé dans la classe,
% à moins que l'option |nonatbib| n'ait été spécifiée au chargement de
% la classe.
%    \begin{macrocode}
\ifthenelse{\boolean{UL@natbib}}{\RequirePackage{natbib}}{}
%    \end{macrocode}
%
% Le support pour les langues autres que l'anglais est offert par le
% paquetage \textsf{babel} --- à moins que l'option |nobabel| n'ait
% été spécifiée au chargement de la classe. Les langues sont passées
% en option de la classe, et non du paquetage. Le paquetage
% \textsf{numprint} est requis par \textsf{babel} pour la définition
% de la commande de mise en forme des nombres |\nombre|.
%    \begin{macrocode}
\ifthenelse{\boolean{UL@babel}}{%
  \RequirePackage{babel}
  \RequirePackage[autolanguage]{numprint}}{}
%    \end{macrocode}
%
% L'insertion du logo de l'Université sur la page titre requiert
% \textsf{graphicx}. On définit également une couleur pour les
% hyperliens dans le document (mais \textsf{hyperref} est chargé dans
% les gabarits afin de demeurer le dernier paquetage chargé; voir la
% \autoref{sec:hyperref}).
%    \begin{macrocode}
\RequirePackage{graphicx}
\RequirePackage{xcolor}
%    \end{macrocode}
%
% La commande |\textcopyright| utilisée sur la page titre requiert le
% paquetage \textsf{textcomp} pour obtenir un beau signe de copyright.
%    \begin{macrocode}
\RequirePackage{textcomp}
%    \end{macrocode}
%
% \subsection{Couleur des hyperliens}
% \label{sec:couleurs}
%
% La classe définit une couleur standard pour les hyperliens, une
% teinte de bleu assez foncée pour être à fois visible en couleur et
% peu contrastante si le document est imprimé en noir et blanc.
%    \begin{macrocode}
\definecolor{ULlinkcolor}{rgb}{0,0,0.3}
%    \end{macrocode}
%
% \subsection{Marges}
%
% Les marges exigées par les normes de présentation de la FESP sont de
% 25~mm partout, sauf 35~mm pour la marge de reliure (gauche pour les
% pages impaires, droite pour les pages paires). Le pied de page est
% placé de sorte que le folio de page se retrouve à 10~mm du bas de la
% page.
%    \begin{macrocode}
\setlrmarginsandblock{35mm}{25mm}{*}
\setulmarginsandblock{25mm}{25mm}{*}
\checkandfixthelayout[nearest]
\setlength{\footskip}{\lowermargin}
\addtolength{\footskip}{-10mm}
%    \end{macrocode}
%
% Comme les thèses et mémoires comportent normalement plusieurs pages
% liminaires, il arrive que des folios (en chiffres romains) dépassent
% dans la marge de droite dans la table des matières. Pour régler ce
% problème, nous augmentont la largeur de la boîte prévue pour les
% imprimer.
%    \begin{macrocode}
\renewcommand{\@pnumwidth}{3em}
\renewcommand{\@tocrmarg}{4em}
%    \end{macrocode}
%
% \subsection{Interligne}
%
% L'espacement entre les lignes est d'un interligne et demi.
% L'espacement «double» entre les paragraphes est fixé à
% |0.5\baselineskip| afin d'en arriver à une disposition agréable à
% l'{\oe}il. Le retrait de première ligne est supprimé puisque plus
% nécessaire suite à l'ajout de l'espacement entre les paragraphes.
%    \begin{macrocode}
\OnehalfSpacing
\setlength{\parskip}{0.5\baselineskip}
\setlength{\parindent}{0em}
%    \end{macrocode}
%
% La table des matières, la liste des tableaux et la liste des figures
% sont composées à interligne simple.
%    \begin{macrocode}
\renewcommand{\tocheadstart}{\SingleSpacing\chapterheadstart}
\renewcommand{\lotheadstart}{\SingleSpacing\chapterheadstart}
\renewcommand{\lofheadstart}{\SingleSpacing\chapterheadstart}
%    \end{macrocode}
%
% \subsection{Entêtes et pieds de page}
%
% Les règles pour les entêtes et pieds de page sont uniformes pour
% tout le document: aucun entête et folio sur le bord extérieur du
% pied de page. On définit un style de page pour ce faire ainsi qu'un
% alias entre le nouveau style |ul| et le style standard |plain|.
% Raison: les premières pages de chapitres utilisent par défaut le
% style |plain|; avec l'alias c'est le style |ul| qui sera activé.
%    \begin{macrocode}
\makepagestyle{ul}
  \makeevenfoot{ul}{\thepage}{}{}
  \makeoddfoot{ul}{}{}{\thepage}
\aliaspagestyle{plain}{ul}
\pagestyle{ul}
%    \end{macrocode}
%
% \subsection{Page titre}
%
% Le code pour traiter et composer la page titre constitue l'essentiel
% de la classe.
%
% \subsubsection{Famille et style de la police de caractères}
%
% La page titre est composée avec la police \textsf{Helvetica}
% (famille \texttt{phv} dans la classification NFSS) dans les
% tailles\footnote{%
%   La police Helvetica produite par {\LaTeX} est plus grande que
%   celle utilisée par Microsoft Word. Pour cette raison, les tailles
%   utilisées dans la classe sont toutes quelques points inférieures à
%   celles des gabarits Word.}%
% et les graisses présentées au \autoref{tab:polices}. La
% déclaration |\fontencoding{T1}| est nécessaire avec {\XeLaTeX} pour
% explicitement charger la même police que sous {\LaTeX}.
%
% \begin{table}
%   \centering
%   \begin{tabular}{ll}
%     \toprule
%     Élément          & Police \\
%     \midrule
%     Titre            & 17~points gras \\
%     Sous-titre       & 14~points gras \\
%     Auteur           & 12~points gras \\
%     Nom du programme & 12~points gras \\
%     Autres éléments  & 12~points normal \\
%     \bottomrule
%   \end{tabular}
%   \caption{Tailles et graisses de la police Helvetica des éléments
%     de la page titre}
%   \label{tab:polices}
% \end{table}
%
%    \begin{macrocode}
\newcommand*{\UL@phvfamily}{\fontencoding{T1}\fontfamily{phv}\selectfont}
%    \end{macrocode}
% Les commandes sélectionnant ces polices sont adaptées selon la
% taille de police choisie pour le document afin d'être toujours
% identiques. Nous utilisons les déclarations de taille de police de
% la classe \textsf{memoir}, présentées au tableau~3.9 de sa
% documentation.
%    \begin{macrocode}
\ifnum\UL@ptsize=10\relax
  \newcommand*{\UL@fonttitle}{\normalfont\huge\bfseries\UL@phvfamily}
  \newcommand*{\UL@fontsubtitle}{\normalfont\LARGE\bfseries\UL@phvfamily}
  \newcommand*{\UL@fontauthor}{\normalfont\Large\bfseries\UL@phvfamily}
  \newcommand*{\UL@fontprogram}{\UL@fontauthor}
  \newcommand*{\UL@fontbase}{\normalfont\Large\UL@phvfamily}
\fi
\ifnum\UL@ptsize=11\relax
  \newcommand*{\UL@fonttitle}{\normalfont\LARGE\bfseries\UL@phvfamily}
  \newcommand*{\UL@fontsubtitle}{\normalfont\Large\bfseries\UL@phvfamily}
  \newcommand*{\UL@fontauthor}{\normalfont\large\bfseries\UL@phvfamily}
  \newcommand*{\UL@fontprogram}{\UL@fontauthor}
  \newcommand*{\UL@fontbase}{\normalfont\large\UL@phvfamily}
\fi
\ifnum\UL@ptsize=12\relax
  \newcommand*{\UL@fonttitle}{\normalfont\Large\bfseries\UL@phvfamily}
  \newcommand*{\UL@fontsubtitle}{\normalfont\large\bfseries\UL@phvfamily}
  \newcommand*{\UL@fontauthor}{\normalfont\normalsize\bfseries\UL@phvfamily}
  \newcommand*{\UL@fontprogram}{\UL@fontauthor}
  \newcommand*{\UL@fontbase}{\normalfont\normalsize\UL@phvfamily}
\fi
%    \end{macrocode}
%
% \subsubsection{Interfaces interne et externe}
%
% Définition des commandes permettant de construire la page titre.
% L'interface utilisateur est basée sur un ensemble de commandes
% internes. On commence par celles-ci.
%    \begin{macrocode}
\newcommand{\UL@maintitle}{}
\newcommand{\UL@subtitle}{}
\newcommand*{\UL@author}{}
\newcommand*{\UL@program}{}
\newcommand*{\UL@year}{}
\newcommand*{\UL@nameother}{}
\newcommand*{\UL@degreeother}{}
\newcommand*{\UL@facUL}{}
\newcommand*{\UL@facother}{}
%    \end{macrocode}
% Puis les commandes visibles pour les utilisateurs, qui redéfinissent
% les commandes internes. Voir la \autoref{sec:commandes} pour leur
% signification. Nous aurons besoin d'une valeur booléenne pour
% retenir si le document a un sous-titre ou non.
%    \begin{macrocode}
\newboolean{UL@hassubtitle}
\newcommand{\titre}[1]{\renewcommand{\UL@maintitle}{#1}}
\newcommand{\soustitre}[1]{%
  \setboolean{UL@hassubtitle}{true}
  \renewcommand{\UL@subtitle}{#1}}
\newcommand*{\auteur}[1]{\renewcommand*{\UL@author}{#1}}
\newcommand*{\annee}[1]{\renewcommand*{\UL@year}{#1}}
\newcommand*{\programme}[1]{\renewcommand*{\UL@program}{#1}}
\newcommand*{\univcotutelle}[1]{\renewcommand*{\UL@nameother}{#1}}
\newcommand*{\gradecotutelle}[1]{\renewcommand*{\UL@degreeother}{#1}}
\newcommand*{\univbidiplomation}[1]{\renewcommand*{\UL@nameother}{#1}}
\newcommand*{\gradebidiplomation}[1]{\renewcommand*{\UL@degreeother}{#1}}
\newcommand{\faculteUL}[1]{\renewcommand*{\UL@facUL}{#1}}
\newcommand*{\faculteUdeS}[1]{\renewcommand*{\UL@facother}{#1}}
\newcommand*{\faculteUQO}[1]{\renewcommand*{\UL@facother}{#1}}
\newcommand*{\faculteUQAC}[1]{\renewcommand*{\UL@facother}{#1}}
%    \end{macrocode}
%
% \subsubsection{Titre et sous-titre}
%
% Outre le nom de l'auteur et la notice de copyright en bas de page,
% la page titre peut être divisée en trois grands blocs:
% \begin{enumerate}
% \item le titre et le sous-titre, le cas échéant;
% \item la description du type de document (thèse, thèse en cotutelle,
%   mémoire, etc.);
% \item les détails sur la ou les facultés, la ou les universités,
%   etc.
% \end{enumerate}
% À ces blocs s'ajoutent quatre grandes catégories de disposition des
% éléments sur la page titre selon le type de thèse ou de mémoire:
% standard; multifacultaire; en cotutelle ou en bidiplomation
% (disposition identique); en extension.
%
% Le titre et le sous-titre peuvent s'étendre sur plus d'une ligne.
% Sans traitement spécial, un long titre ou sous-titre aurait pour
% impact de décaler vers le bas tous les autres éléments de la page
% titre. Pour contrer ce phénomène, nous devrons mesurer la hauteur du
% titre et du sous-titre pour ensuite ajuster en conséquence la
% distance entre ce bloc et la mention du type de document.
%
% \begin{macro}{\UL@measuretitle}
%   On place le titre et le sous-titre centrés dans des boîtes
%   |\UL@titlebox| et |\UL@subtitlebox|. La commande
%   |\UL@measuretitle| permettra de mesurer leur hauteur lorsque le
%   titre sera créé avec |\pagetitre|, plus loin. Un espacement
%   vertical d'un demi interligne est ajouté entre le titre et le
%   sous-titre, le cas échéant.
%    \begin{macrocode}
\newsavebox{\UL@titlebox}
\newsavebox{\UL@subtitlebox}
\newlength{\UL@titleboxtotht}
\newlength{\UL@subtitleboxtotht}
\newcommand{\UL@measuretitle}{%
  \setbox\UL@titlebox=\vbox{%
    \centering\UL@fonttitle\UL@maintitle}
  \setlength{\UL@titleboxtotht}{%
    \dimexpr\ht\UL@titlebox+\dp\UL@titlebox}
  \ifthenelse{\boolean{UL@hassubtitle}}{%
    \setbox\UL@subtitlebox=\vbox{%
      \centering\vspace*{0.5\baselineskip}\UL@fontsubtitle\UL@subtitle}
    \setlength{\UL@subtitleboxtotht}{%
      \dimexpr\ht\UL@subtitlebox+\dp\UL@subtitlebox}}{}}
%    \end{macrocode}
% \end{macro}
%
% \subsubsection{Type de document}
%
% \begin{macro}{\Ul@docid}
%   La commande |\Ul@docid| prépare la mention du type de document. La
%   thèse ou le mémoire en cotutelle ou en bidiplomation requiert un
%   traitement différent puisque le programme d'étude apparaît
%   immédiatement sous la mention.
%    \begin{macrocode}
\newcommand{\UL@docid}{%
  {\UL@fontprogram\UL@typeofdoc\par
  \ifnum\UL@typenum=2 \UL@program\par \fi}}
%    \end{macrocode}
% \end{macro}
%
% \subsubsection{Détails sur les facultés et universités d'attache}
%
% \begin{macro}{\Ul@details}
%   La commande |\Ul@details| est la plus complexe puisque la
%   disposition des informations additionnelles sur le document varie
%   beaucoup selon le type de thèse ou de mémoire. Tel qu'expliqué à
%   la \autoref{sec:utilisation:options}, certains types de grade
%   requièrent expressément que certaines informations soient
%   fournies. Si un élément d'information manque, un avertissement
%   est émis.
%    \begin{macrocode}
\newcommand{\UL@details}{%
  \ifcase\UL@typenum\relax% 0 standard
    \vspace{96pt}
    {\UL@fontprogram\UL@program}\par
    \UL@degree\par
    \vspace{112pt}
    Qu\'ebec, Canada\par
  \or%                      1 multifacultaire
    \vspace{96pt}
    {\UL@fontprogram\UL@program}\par
    \UL@degree\par
    \vspace{36pt}
    \ifx\UL@facUL\empty
      \ClassWarningNoLine{ulthese}{UL faculty names missing.}
    \else
      \UL@facUL\par
    \fi
    \vspace{48pt}
    Qu\'ebec, Canada\par
  \or%                      2 cotutelle et bidiplomation
    \vspace{72pt}
    Universit\'e Laval\par Qu\'ebec, Canada\par
    \UL@degree\par
    \vspace{\baselineskip} et\par \vspace{\baselineskip}
    \ifx\UL@nameother\empty
      \ClassWarningNoLine{ulthese}{Other university name and location missing}
    \else
      \UL@nameother\par
    \fi
    \ifx\UL@degreeother\empty
      \ClassWarningNoLine{ulthese}{Other university degree missing}
    \else
      \UL@degreeother\par
    \fi
  \or%                      3 extension
    \vspace{48pt}
    {\UL@fontprogram\UL@program\ de l'Universit\'e Laval\par
      \ifthenelse{\boolean{UL@isprogmasc}}{offert}{offerte}
      en extension \`a l'\UL@extensionat}\par
    \vspace{36pt}
    \UL@degree\par
    \vspace{36pt}
    \ifx\UL@facother\empty
      \ClassWarningNoLine{ulthese}{Other university faculty name missing}
    \else
      \UL@facother\par
    \fi
    \UL@extensionat\par
    \UL@extensionloc\par
    \vspace{\baselineskip}
    \ifx\UL@facUL\empty
      \ClassWarningNoLine{ulthese}{UL faculty name missing}
    \else
      \UL@facUL\par
    \fi
    Universit\'e Laval\par Qu\'ebec, Canada\par
  \fi}
%    \end{macrocode}
% \end{macro}
%
% \subsubsection{Conception de la page titre}
%
% \begin{macro}{\pagetitre}
%   On doit rétablir sur la page titre l'interligne simple et
%   l'espacement nul entre les paragraphes (|\parskip|). Ensuite, on
%   doit ajuster la distance entre le bloc de titre et le type de
%   document (|\UL@docidspacing|) et celle entre ce dernier et le nom
%   de l'auteur (|\UL@authorspacing|). Cela fait en sorte que les
%   éléments de la page titre se retrouvent (presque) toujours au même
%   endroit sur la page. Une distance minimale d'un interligne est
%   conservée entre le bloc de titre et le type de document
%   (précaution nécessaire pour l'éventuel cas d'un bloc de titre
%   s'étendant sur plusieurs lignes).
%
%   La page titre des thèses et mémoires en cotutelle, en
%   bidiplomation ou en extension ne comporte pas de logo de
%   l'Université Laval.
%
%   Le nom de l'auteur et la notice de copyright sont insérées
%   directement dans le code de la commande |\pagetitre|.
%    \begin{macrocode}
\newlength{\UL@docidspacing}
\setlength{\UL@docidspacing}{82pt}
\newlength{\UL@authorspacing}
\setlength{\UL@authorspacing}{72pt}
\newcommand{\pagetitre}{{%
    \clearpage
    \thispagestyle{empty}
    \SingleSpacing\setlength{\parskip}{0pt}
    \centering
    \UL@fontbase
    \UL@measuretitle
    \addtolength{\UL@docidspacing}{-\UL@titleboxtotht}
    \addtolength{\UL@docidspacing}{-\UL@subtitleboxtotht}
    \ifdim\UL@docidspacing<\baselineskip\relax
      \setlength{\UL@docidspacing}{\baselineskip}
      \addtolength{\UL@authorspacing}{-\baselineskip}
    \fi
    \ifnum\UL@typenum>1\relax
      \vspace*{0pt}\par
    \else
      \includegraphics[height=15mm,keepaspectratio=true]{ul_p}\par
    \fi
    \vspace{82pt}
    \box\UL@titlebox
    \box\UL@subtitlebox
    \vspace{\UL@docidspacing}
    \UL@docid
    \vspace{\UL@authorspacing}
    {\UL@fontauthor\UL@author}\par
    \UL@details
    \vfill
    {\textcopyright} \UL@author, \UL@year\par
    \cleardoublepage}}
%    \end{macrocode}
% \end{macro}
%
% \subsection{Listes des figures et des tableaux}
%
% \begin{macro}{\listfigurename}
%   Le paquetage \textsf{babel} définit comme titre pour la liste des
%   figures «Table des figures», alors que la liste des tableaux est
%   «Liste des tableaux». Pour une plus grande symétrie, la classe
%   redéfinit le titre correspondant à |\listoffigures|. La commande
%   |\addto| est nécessaire pour éviter que \textsf{babel} redéfinisse
%   le titre à |\begin{document}|.
%    \begin{macrocode}
\ifthenelse{\boolean{UL@babel}}{%
  \addto\captionsfrench{\renewcommand{\listfigurename}{Liste des figures}}}{}
%    \end{macrocode}
%   Si \textsf{babel} n'est pas chargé, ce sera à l'utilisateur de faire
%   une correction équivalente. Avec \textsf{polyglossia}, la commande à
%   insérer dans l'entête du document est la même que ci-dessus.
% \end{macro}
%
% \subsection{Dédicace et épigraphe}
%
% La dédicace et l'épigraphe sont mises en forme avec la commande
% |\epigraph| de \textsf{memoir}.
% \begin{macro}{\dedicace}
%   La dédicace est une épigraphe simplifiée placée seule sur une
%   page, alignée à droite à une dizaine de lignes de la marge
%   supérieure, sans auteur ou source et sans ligne de démarcation.
%    \begin{macrocode}
\newcommand{\dedicace}[1]{{%
    \clearpage
    \pagestyle{empty}
    \setlength{\beforeepigraphskip}{10\baselineskip}
    \setlength{\epigraphrule}{0pt}
    \epigraphtextposition{flushright}
    \mbox{}\epigraph{\itshape #1}{}}}
%    \end{macrocode}
% \end{macro}
% \begin{macro}{\epigraphe}
%   L'épigraphe de début de document est placée seule sur une page à
%   une dizaine de lignes de la marge supérieure. Pour le reste, on
%   s'en remet à la commande |\epigraph| de \textsf{memoir}.
%    \begin{macrocode}
\newcommand{\epigraphe}[2]{{%
    \clearpage
    \pagestyle{empty}
    \setlength{\beforeepigraphskip}{10\baselineskip}
    \mbox{}\epigraph{#1}{#2}}}
%    \end{macrocode}
% \end{macro}
%
% \subsection{Citations}
%
% \begin{environment}{quote}
%   La classe redéfinit l'environnement |quote| de \textsf{memoir}
%   afin que le texte des citations se trouve en retrait de 10~mm à
%   gauche et à droite, conformément aux règles de présentation de la
%   FESP.
%    \begin{macrocode}
\renewenvironment{quote}{%
  \list{}{\rightmargin 10mm \leftmargin 10mm}%
  \item[]}{\endlist}
%    \end{macrocode}
%   \end{environment}
%
% \begin{environment}{quotation}
%   Il en va de même de l'environnement |quotation|. Cependant, cet
%   environnement passe également à l'interligne simple et la classe
%   ajuste l'espacement vertical entre les paragraphes afin que
%   ceux-ci soient bien distincts les uns des autres tout en demeurant
%   raisonnablement compacts. Cet espacement est ici fixé à 6~points.
%    \begin{macrocode}
\renewenvironment{quotation}{%
  \list{}{%
    \SingleSpacing
    \listparindent 0em
    \itemindent    \listparindent
    \leftmargin    10mm
    \rightmargin   \leftmargin
    \parsep        6\p@ \@plus\p@}%
  \item[]}{\endlist}
%    \end{macrocode}
% \end{environment}
%
% \subsection{Numérotation des divisions du document}
%
% Par défaut, \textsf{memoir} numérote les divisions du document
% seulement jusqu'au niveau des sections. La classe étend la
% numérotation aux sous-sections.
%    \begin{macrocode}
\setsecnumdepth{subsection}
%</class>
%    \end{macrocode}
% ^^A Fin du code de la classe
%
% \begin{thebibliography}{1}
% \providecommand{\bibnamefont}[1]{#1}
% \providecommand{\selectlanguage}[1]{\relax}
% \bibitem[{Malo(1996)}]{Malo:1996}
% \bibnamefont{Malo}, M. 1996, {\selectlanguage{francais}\emph{Guide de la
%   communication écrite au cégep, à l'université et en entreprise}}, Québec
%   Amérique. ISBN~978-2-8903-7875-9.
% \end{thebibliography}
%
% \Finale
%
% \iffalse
% ^^A Gabarits du document maître
%<*gabarit>
%<phd&standard>%% GABARIT POUR THÈSE STANDARD
%<phd&mesure>%% GABARIT POUR THÈSE SUR MESURE
%<phd&multifac>%% GABARIT POUR THÈSE MULTIFACULTAIRE
%<phd&cotutelle>%% GABARIT POUR THÈSE EN COTUTELLE
%<phd&UdeS>%% GABARIT POUR THÈSE EN EXTENSION À L'UNIVERSITÉ DE SHERBROOKE
%<phd&UQO>%% GABARIT POUR THÈSE EN EXTENSION À L'UQO
%<m&standard>%% GABARIT POUR MÉMOIRE STANDARD
%<m&mesure>%% GABARIT POUR MÉMOIRE SUR MESURE
%<m&bidiplomation>%% GABARIT POUR MÉMOIRE EN BIDIPLOMATION
%<m&UQAC>%% GABARIT POUR MÉMOIRE EN EXTENSION À L'UQAC
%%
%% Consulter la documentation de la classe ulthese pour une
%% description détaillée de la classe, de ce gabarit et des options
%% disponibles.
%%
%% [Ne pas hésiter à supprimer les commentaires après les avoir lus.]
%%
%% Déclaration de la classe avec le type de grade
%<phd>%%   [l'un de PhD, LLD, DPsy, DThP]
%<m>%%   [l'un de MSc, LLM, MA, MMus, MServSoc, MScGeogr, MATDR]
%% et les langues les plus courantes. Le français sera la langue par
%% défaut du document.
%<phd&(standard|mesure)>\documentclass[PhD,english,francais]{ulthese}
%<phd&multifac>\documentclass[PhD,multifacultaire,english,francais]{ulthese}
%<phd&cotutelle>\documentclass[PhD,cotutelle,english,francais]{ulthese}
%<phd&UdeS>\documentclass[PhD,extensionUdeS,english,francais]{ulthese}
%<phd&UQO>\documentclass[PhD,extensionUQO,english,francais]{ulthese}
%<m&(standard|mesure)>\documentclass[MSc,english,francais]{ulthese}
%<m&bidiplomation>\documentclass[MA,bidiplomation,english,francais]{ulthese}
%<m&UQAC>\documentclass[MSc,extensionUQAC,english,francais]{ulthese}
  %% Encodage utilisé pour les caractères accentués dans les fichiers
  %% source du document. Les gabarits sont encodés en UTF-8. Inutile avec
  %% XeLaTeX, qui gère Unicode nativement.
  \ifxetex\else \usepackage[utf8]{inputenc} \fi

  %% Charger ici les autres paquetages nécessaires pour le document.
  %% Quelques exemples; décommenter au besoin.
  %\usepackage{amsmath}          % recommandé pour les mathématiques
  %\usepackage{icomma}           % gestion de la virgule dans les nombres

  %% Utilisation d'une autre police de caractères pour le document.
  %% - Sous LaTeX
  %\usepackage{mathpazo}         % texte et mathématiques en Palatino
  %\usepackage{mathptmx}         % texte et mathématiques en Times
  %% - Sous XeLaTeX
  %\setmainfont{TeX Gyre Pagella}      % texte en Pagella (Palatino)
  %\setmathfont{TeX Gyre Pagella Math} % mathématiques en Pagella (Palatino)
  %\setmainfont{TeX Gyre Termes}       % texte en Termes (Times)
  %\setmathfont{TeX Gyre Termes Math}  % mathématiques en Termes (Times)

  %% Gestion des hyperliens dans le document. S'assurer que hyperref
  %% est le dernier paquetage chargé.
  \usepackage{hyperref}
  \hypersetup{colorlinks,allcolors=ULlinkcolor}

  %% Options de mise en forme du mode français de babel. Consulter la
  %% documentation du paquetage babel pour les options disponibles.
  %% Désactiver (effacer ou mettre en commentaire) si l'option
  %% 'nobabel' est spécifiée au chargement de la classe.
  \frenchbsetup{%
    CompactItemize=false,         % ne pas compacter les listes
    ThinSpaceInFrenchNumbers=true % espace fine dans les nombres
  }

  %% Style de la bibliographie.
  \bibliographystyle{}

  %% Déclarations de la page titre. Remplacer les éléments entre < >.
  %% Supprimer les caractères < >. Couper un long titre ou un long
  %% sous-titre manuellement avec \\.
  \titre{<Titre principal>}
  % \titre{Ceci est un exemple de long titre \\
  %   avec saut de ligne manuel}
  % \soustitre{Sous-titre le cas échéant}
  % \soustitre{Ceci est un exemple de long sous-titre \\
  %   avec saut de ligne manuel}
  \auteur{<Prénom Nom>}
%<phd&(standard|multifac|cotutelle)>  \programme{Doctorat en <discipline> -- <majeure, s'il y a lieu>}
%<phd&mesure>  \programme{Doctorat sur mesure en <discipline> -- <majeure, s'il y a lieu>}
%<phd&(UdeS|UQO)>  \programme{Doctorat en <discipline>}
%<m&standard>  \programme{Maîtrise en <discipline> -- <majeure, s'il y a lieu>}
%<m&mesure>  \programme{Maîtrise sur mesure en <discipline> -- <majeure, s'il y a lieu>}
%<m&bidiplomation>  \programme{Maîtrise en études anciennes}
%<m&UQAC>  \programme{Maîtrise en <discipline>}
  \annee{<20xx>}
%<cotutelle>  \univcotutelle{<Université de cotutelle> \\ <Ville>, <Pays>}
%<bidiplomation>  \univcotutelle{<Université de bidiplomation> \\ <Ville>, <Pays>}
%<cotutelle|bidiplomation>  \gradecotutelle{<Nom du grade> (<sigle du grade>)}
%<multifac>  \faculteUL{<Faculté 1> \\ <Faculté 2>}
%<UdeS|UQO|UQAC>  \faculteUL{<Nom de la faculté UL>}
%<UdeS>  \faculteUdeS{<Nom de la faculté UdeS>}
%<UQO>  \faculteUQO{<Nom de la faculté UQO>}
%<UQAC>  \faculteUQAC{<Nom de la faculté UQAC>}

\begin{document}

\frontmatter                    % pages liminaires

\pagetitre                      % production de la page titre

\chapter*{Résumé}
\phantomsection\addcontentsline{toc}{chapter}{Résumé}

\begin{otherlanguage*}{francais}

    Pour assurer une navigation sécuritaire et efficace, les robots mobiles reposent grandement sur l'utilisation des capteurs embarqués. L'un des capteurs qui est de plus en plus utilisé pour la navigation des robots est le \emph{Light Detection And Ranging} (LiDAR). Il produit de l'information riche sur la structure tridimensionnelle de l'environnement du robot et ce, indépendemment des conditions lumineuses. Bien que les recherches récentes montrent une amélioration des performances de navigation pour des problèmes contraints comme la conduite urbaine dans les conditions ensoleillés de la Californie, faire face à des environnements non-structurés complexes ou des conditions météorologiques difficiles reste problématique. Dans ce mémoire, nous présentons une analyse de l'influence de telles conditions sur la navigation basée sur les LiDARs. Notre première contribution est d'évaluer comment quatre LiDARs couramment utilisés (Velodyne HDL-32E, SICK LMS151, SICK LMS200 and Hokuyo UTM-30LX-EW) sont affectés par les flocons de neige durant des tempêtes de neige. Nous créons un nouvel ensemble de données en faisant l'acquisition de données avec les quatre capteurs simultanément durant six précipitations de neige. Une analyse statistique de ces ensembles de données indique que leurs mesures peuvent être modélisés de manière probalistique, permettant l'utilisation d'un système bayésien pour améliorer la robustesse. De plus, nous observons l'évolution temporelle de l'impact de la neige durant ces précipitations, et caractérisons la sensibilité de chaque capteur. Finalement, nous concluons que les précipitations de neige ont peu d'influence au-delà de \SI{10}{\meter}. Notre seconde contribution est d'évaluer l'impact de structures tridimensionnelles complexes présentes en forêt sur les performances des algorithmes de navigation. Étant donné que l'habileté à reconnaître des endroits visités est utile à plusieurs problèmes de navigation, nous avons choisi d'effectuer nos expériences avec un algorithme de l'état de l'art en reconnaissance d'endroits. Nous avons acquis des données dans un environnement extérieur structuré et en forêt ainsi qu'avec deux LiDARs (LMS151, HDL-32E). Cela permet d'évaluer l'influence de l'environnement sur les performances de reconnaissance d'endroits, mais aussi celle du capteur chosi. Notre hypothèse est que le plus proche deux balayages laser sont l'un de l'autre, le plus haut le score (i.e. la croyance que ces balayages laser proviennent du même endroit) sera, mais modulé par le niveau de complexité de l'environnement. Nos expériences confirment que la forêt, avec ses réseaux de branches compliqués et son feuillage, produit plus de données abérantes et induit un chute plus rapide des scores en fonction de la distance que pour les environnements structurés, présent entre des batîments. De façon similaire, les scores pour le Velodyne chutes légèrement plus rapidement en fonction de la distance que pour le SICK. Notre conclusion finale est que, les environnements complexes étudiés impactent négativement les performances de navigation basée sur les LiDARs, ce qui devrait être considéré pour développer des algorithmes de navigation robustes.


\end{otherlanguage*}
                % résumé français
\chapter*{Abstract}
\phantomsection\addcontentsline{toc}{chapter}{Abstract}

\begin{otherlanguage*}{english}
Text of English abstract.
\end{otherlanguage*}
              % résumé anglais
\cleardoublepage

\tableofcontents                % production de la TdM
\cleardoublepage

\listoftables                   % production de la liste des tableaux
\cleardoublepage

\listoffigures                  % production de la liste des figures
\cleardoublepage

\dedicace{Dédicace si désiré}
\cleardoublepage

\epigraphe{Texte de l'épigraphe}{Source ou auteur}
\cleardoublepage

\include{remerciements}         % remerciements
\include{avantpropos}           % avant-propos

\mainmatter                     % corps du document

\include{introduction}          % introduction
\include{chapitre1}             % chapitre 1
\include{chapitre2}             % chapitre 2, etc.
\section{Discussion and Conclusion}

In this paper, we explored the impact of falling snow on the usability of 4 commonly deployed LiDARs in the context of autonomous driving vehicles. To this end, we collected data during 6 snowstorms in the winter of 2015. Upon analysis, we found that the SICK LMS200 was the most sensitive LiDAR, having a peak average rate of up to 15~\% of echoes coming from falling snow. Meanwhile, all 3 others never exceeded 1~\%. We also presented a simple probabilistic model to take into account the effect of the range on snowflakes interference. Based on a histogram analysis, we concluded that for our experimental setup, this model can be approximated by a log-normal distribution. Most importantly, our data indicate that the impact of snowflakes on LiDAR beyond a range of \SI{10}{\meter} is very limited. 

%A significant side benefit of the more sensitive sensors (SICK LMS-200 or the $1^{st}$ echo of the Hokuyo UTM-30LX-EW) is that they are capable of recording the temporal evolution of a snowstorm, at a fine-grained level. This could be helpful in developing temporal models of snowstorms, to be used in a vehicle simulator.
%This is something  not possible with traditional snow measuring equipments, which can only report accumulation over long period of times.

However, a number of questions remains to explore. For example, as the LiDAR beam travels through the falling snow, its intensity will diminish. Since the maximum range of a LiDAR is heavily related to this beam intensity, we expect the maximum range to be affected during snowstorms. In our setup, we have not witnessed this issue, indicating that this effect probably happens beyond our maximum distance of \SI{20}{\meter}. Another aspect to be investigated is the relationship between the returned intensities and the surface type (ground or snowflakes). Also, because of the shielding effect of the building, very few snowflakes were present at close range; It might be the case that at closer range, a snowflake might be detected at more than one angle, effectively occluding small targets. Moreover, we have not investigated the impact on the measurement noise for the snowy ground surface in the presence of falling snow. Finally, it would be interesting to mount these LiDARs on a moving vehicle to investigate the impact of the vehicle velocity on the sensing behavior.


% We would like to understand the impact of rain on these LiDAR.
%We also have one dataset taken during a rain shower.
% Also look at how noise was affected.
%Another key aspect would be to estimate the impact of sunlight on the measurements, (this has been done by F. Pomerleau in some sense. We have other datasets that were collected in overcast and sunny days, but without any falling snow.
            % conclusion

\appendix                       % annexes le cas échéant

\include{annexe}                % annexe A

\bibliography{}                 % production de la bibliographie

\end{document}
%</gabarit>
%
% ^^A Gabarits des parties du document
%<*resume>
\chapter*{Résumé}                      % ne pas numéroter
\phantomsection\addcontentsline{toc}{chapter}{Résumé} % inclure dans TdM

\begin{otherlanguage*}{francais}
  Texte du résumé en français.
\end{otherlanguage*}
%</resume>
%
%<*abstract>
\chapter*{Abstract}                      % ne pas numéroter
\phantomsection\addcontentsline{toc}{chapter}{Abstract} % inclure dans TdM

\begin{otherlanguage*}{english}
  Text of English abstract.
\end{otherlanguage*}
%</abstract>
%
%<*remerciements>
\chapter*{Remerciements}         % ne pas numéroter
\phantomsection\addcontentsline{toc}{chapter}{Remerciements} % inclure dans TdM

Texte des remerciements en prose.
%</remerciements>
%
%<*avantpropos>
\chapter*{Avant-propos}         % ne pas numéroter
\phantomsection\addcontentsline{toc}{chapter}{Avant-propos} % inclure dans TdM

L'avant-propos est surtout nécessaire pour une thèse par article.
%</avantpropos>
%
%<*introduction>
\chapter*{Introduction}         % ne pas numéroter
\phantomsection\addcontentsline{toc}{chapter}{Introduction} % inclure dans TdM

Une thèse ou un mémoire devrait normalement débuter par une
introduction. Celle-ci est traitée comme un chapitre normal, sauf
qu'elle n'est pas numérotée.
%</introduction>
%
%<*chapitre>
\chapter{Titre du chapitre}     % numéroté

Texte du chapitre.
%</chapitre>
%
%<*conclusion>
\chapter*{Conclusion}         % ne pas numéroter
\phantomsection\addcontentsline{toc}{chapter}{Conclusion} % dans TdM

Une thèse ou un mémoire devrait normalement se terminer par une
conclusion, placée avant les annexes, le cas échéant. Celle-ci est
traitée comme un chapitre normal, sauf qu'elle n'est pas numérotée.
%</conclusion>
%
%<*annexe>
\chapter{Titre de l'annexe}     % numérotée

Texte de l'annexe.
%</annexe>
% Local Variables:
% mode: doctex
% coding: utf-8
% TeX-master: t
% End:
% \fi
